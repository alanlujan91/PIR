
\hypertarget{cross-sectional-heterogeneity}{%
\subsection{Cross-sectional
heterogeneity}\label{cross-sectional-heterogeneity}}

This section inspects some basic cross-sectional patterns of the subject
moments of labor income. In Figure \ref{fig:histmoms}, I plot the
distribution of unexplained residuals of perceived income risks both in nominal and real terms after controlling for observable individual characteristics including age, age polynomial, gender, education, type of work, and time fixed effect, respectively.

There remains a sizable dispersion in perceived income risks. In both nominal
and real terms, the distribution is right-skewed with a long tail.
Specifically, most of the workers have perceived a variance of nominal
earning growth ranging from zero to \(20\) (a standard-deviation
equivalence of \(4-4.5\%\) income growth a year). But in the tail, some
of the workers perceive risks to be as high as \(7-8\%\) standard
deviation a year. To have a better sense of how large the risk is,
consider a median individual in our sample, who has an expected earnings
growth of \(2.4\%\), and a perceived risk of \(1\%\) standard deviation.
This implies by no means negligible earning risk.
\footnote{In the appendix, I also include histograms of expected income growth and subjective skewness, which show intuitive patterns such as nominal rigidity. Besides, about half of the sample exhibits non-zero skewness in their subjective distribution, indicating asymmetric upper/lower tail risks.}

\begin{center}
[FIGURE \ref{fig:histmoms} HERE]
\end{center}

How are perceived income risks different along important dimensions of observable individual characteristics? Empirical estimates of income risks of different demographic
groups from microdata have been rare but not non-existant\footnote{For instance, \cite{meghir2004income} estimated that the high-education group is faced with higher income risks than the low-education group.  This is further confirmed by my estimation using SIPP.  In addition, \cite{sabelhaus2010great, bloom2018great} documented that income risks decrease with age, and vary with current income level in a non-monotonic U-shape.}. It is worth asking if subjective risk perceptions exhibit similar between-group differences. This helps evaluate to what extent heterogeneity in risk perceptions partly reflects the actual differences in income risks.

 Figure
\ref{fig:age_compare} plots both perceived and realized income volatility over the life-cycle. In order to control for the differences in risks between gender and education, I calculate the average within gender and education groups. It is clear that the subjective risk perceptions decline over the life cycle, consistent with the estimated risk from realizations of income. It is important to notice, however, in principle, the reasons for which subjective risk perceptions decline as one age may not be exactly the same as the one for the same pattern of the actual profile. For instance,  as one accumulates experience over time, it may also reduce the subjective uncertainty about the income dynamics of themselves. 

\begin{center}
 [FIGURE \ref{fig:age_compare} HERE]
 \end{center}
 
Another important question is how income risk perceptions correlate with the
realized labor income. This is unclear in theory because it could depend on both the true
income process and the perception formation. For a subsample of around 4000 observations, SCE surveys the annual earning of the respondent along with their risk perceptions. I group individuals into 10 groups based on their reported earning (within the same time) and plot the average risk perceptions against the decile rank in Figure \ref{fig:barplot_byinc}. Perceived risks decline as one's earnings increase.  This is not exactly consistent with the uptick in income risks for the highest income group, as documented by \cite{bloom2018great} using tax records of income. The most likely explanation is that the small sample I used from SCE does not cover actual top earners. The average annual earning of the top income group is between \$45,000 and \$120,000 in our sample.  

\begin{center}
 [FIGURE \ref{fig:barplot_byinc} HERE]
\end{center}

