\hypertarget{model_pe}{%
\section{Risk perceptions and wealth inequality}\label{model_pe}}

\subsection{An overlapping generation model}


I set up a standard incomplete market/life-cycle model without aggregate risks. It is built on \cite{huggett1996wealth}, except for  a more realistic income risk profile and economic environment a la \cite{krueger2016macroeconomics,carroll2017distribution}. Then, I add one single deviation from the benchmark to allow for the subjective income risk perceptions. 

In each period, a continuum of agents is born. Each agent $i$ lives for $L$ and works for $T$ ($T\leq L$) periods since entering the labor market, during which he/she earns stochastic labor income \(y_\tau\) at the
work-age of \(\tau\). After retiring at age of \(T\), the agent lives
for another \(L-T\) periods of life and receive social security benefits. We assume away aggregate risks in the benchmark model, therefore there is no need to treat calendar time $t$ from working age $\tau$ as two separate state variables, hence we suppress time script $t$. All shocks are idiosyncratic, or to put it differently, specific to the individual $i$. 


\subsubsection{Consumer's problem}

The consumer chooses the whole future consumption path to maximize
expected life-long utility, under a discount factor $\beta$ and constant survival probability $(1-D)$. 

\begin{equation}
\textrm{max}\quad  \mathbb{E}\left[\sum^{\tau=L-1}_{\tau=0}(1-D)^\tau\beta^\tau u(c_{i,\tau})\right] 
\end{equation}


where $c_{i,\tau}$ represents consumption at the work-age of $\tau$. The
felicity function $u(c)$ takes a standard CRRA form with relative risk
aversion of $\rho$: $u(c) = \frac{c^{1-\rho}}{1-\rho}$.  \footnote{There is are bequest motive and
preference-shifter along life cycle, but these features can be easily incorporated.}


 Denote total cash in hand at the beginning of the period
\(\tau\) as $m_{i,\tau}$, the end-of-period saving in period $\tau$ after consumption as $a_{i,\tau}$, the bank balance in period $\tau$ as $b_{i,\tau}$. Also, assume 
$R$ is the gross real interest factor. The consumer starts with some positive bank balance in the first period of life,  $b_1$, which this paper assumes to be from a lump-sum accidental bequest from the deceased population each period. The household makes consumption and saving decisions subject to the following intertemporal budget constraint.

\begin{equation}
\begin{split}
& a_{i,\tau} = m_{i,\tau} - c_{i,\tau} \\
& b_{i,\tau+1} = a_{i,\tau} R  \\
& m_{i,\tau+1}   = b_{i,\tau+1}+(1-\lambda)y_{i,\tau+1}\\
& a_{i,\tau} \geq 0 
\end{split}
\end{equation}

In addition, I impose an external zero borrowing constraint.  Without the external borrowing constraint, the agent will still self-imposed a lower bound for $\under a_\tau$ to avoid the extremely painful zero consumption next period in the case of the worst draw of income shocks. 



\subsubsection{Income process}

Each agent receives stochastic labor income during working age from $\tau=0$ to $\tau=T$ and receives social security benefit after retirement. The income processes in both sub-periods can be defined in a generic manner as described below. In particular, it is assumed to follow a slight variant of the standard permanent/transitory income process used in the literature\footnote{\cite{carroll2017distribution}, \cite{kaplan2018microeconomic}, etc.} by allowing the possibility of persistent unemployment risks. Specifically, $y_{i,\tau}$ is a multiplication of idiosyncratic labor productivity $n_{i,\tau}$ and the economy-wide wage rate $W$. The former consists of one permanent component $p_{i,\tau}$ and one potentially persistent or transitory $\xi_{i,\tau}$. The aggregate wage is to be determined by the general equilibrium. \footnote{In the presence of aggregate risk, we need to allow $W$ being time-varying and this also means we need to be explicit about the difference between the calendar year and working age.} 

\begin{equation}
\begin{split}
y_{i,\tau} = n_{i,\tau}W \\ 
n_{i,\tau} = p_{i,\tau}\xi_{i,\tau}
\end{split}
\end{equation}

During the work, the permanent income component is subject to a mean-one white-noise shock $\psi$ in each period and grows according to a deterministic life-cycle profile governed by $G_\tau$, which usually follows a hump-shape according to existing estimates. \citep{gourinchas2002consumption}

\begin{equation}
\begin{split}
& p_{i,\tau} = G_\tau p_{i,\tau-1}\psi_{i,\tau} \\
& log (\psi_{i,\tau}) \sim N(-\frac{\sigma^2_{\psi}}{2},\sigma^2_{\psi}) \quad \forall \tau \leq T \\
\end{split}
\end{equation}

The persistent/transitory shock $\xi_{i,\tau}$ takes different values depending on the transitory or persistent state of unemployment following a Markov process.\footnote{This formulation follows \cite{krueger2016macroeconomics}.}  

\begin{equation}
\begin{split}
& \xi_{i,\tau} =   \left\{
\begin{array}{ll}
 \theta_{i,\tau} \quad \text{if} \quad \nu_{i,\tau} =e \quad \& \quad  \tau \leq T \\
      \zeta \quad \text{if} \quad \nu_{i,\tau} = u \quad \& \quad \tau \leq T  \\
      \mathbb{S} \quad \text{if}  \quad \tau > T
\end{array} \right. \\
& log(\theta_{i,\tau}) \sim N(-\frac{\sigma^2_\theta}{2},\sigma^2_\theta)
\end{split}
\end{equation}

where $\zeta$ is the replacement ratio of the unemployment insurance and $\theta_{i,\tau}$ is the i.i.d. mean-one white noise shock to the transitory component of the income conditional on staying employed. Notice that the process above also embodies the income process after retirement after $\tau=T$. The agent receives social security with replacement ratio $\mathbb{S}$ and proportional to her permanent income and aggregate wage rate. Therefore, the effective pension benefit received is $\mathbb{S}p_{i,\tau}W$. I assume that the permanent income component just follows the determinist path without additional stochastic shocks. 

During work age of any individual $i$, the transition matrix between unemployment ($\nu_{i,\tau}=u$) and employment ($\nu_{i,\tau} = e$) is the following.

\begin{equation}
\pi(\nu_{\tau+1}|\nu_{\tau})= 
    \begin{bmatrix} 
	\mho & 1-\mho  \\
	1-E & E
	\end{bmatrix}
	\quad
\end{equation}

In general, this assumption implies some degree of the persistence of unemployment risks, but it conveniently nests the special case where the unemployment risk is purely transitory when $\mho = 1-E$, meaning the probability of unemployment is not dependent on the current status. 


Unemployment risks are idiosyncratic, hence by the law of large numbers, the fraction of the population being unemployed and employed at each age, denoted by $\Pi^\mho_\tau$ and $\Pi^E_\tau$, respectively, are essentially deterministic and does not depend on age.  

Notice that in the benchmark model laid out here, I assume the all parameters of income risks $\sigma_\psi$, $\sigma_\theta$, $\mho$, and $E$ to be age-invariant (equivalent to time-independent in this setting). By doing this, I avoid making explicit assumptions on the stochastic process of income risks. This is a common practice in the incomplete market macro literature since    \cite{gourinchas2002consumption} and \cite{cagetti2003wealth}. It is also not fundamentally different from assuming a deterministic age-specific risk profile, as in some variants of the models with the life-cycle component. \footnote{See \cite{carroll2017distribution} and other examples.} I relax this assumption by considering an explicitly specified stochastic process of the income volatility a la GARCH in the extension of the model (See the Appendix). 

\subsubsection{Value function and consumption policy}

The following value function characterizes the  consumer's problem.

\begin{equation}
\begin{split}
V_{\tau}(\nu_{i,\tau}, m_{i,\tau}, p_{i,\tau}) = \underset{\{c_{i,\tau},a_{i,\tau}\}}{\textrm{max}} \quad u(c_{i,\tau}) + (1-D)\beta \mathbb{E}_{\tau}\left[V_{\tau+1}(\nu_{i,\tau},m_{i,\tau+1}, p_{i,\tau+1})\right] 
\end{split}
\end{equation}

where the three state variables for the agents are current employment status $\nu_{i,\tau}$,  total cash in hand $m_{i,\tau}$ and permanent income $p_{i,\tau}$.  $\nu_{i,\tau}$ drops from the state variables in the special case of purely transitory unemployment shock ($\mho = 1-E$).\footnote{Another trick used in the literature to reduce the number of state variables is to normalize the value function by permanent income level $p_\tau$, so that it drops from the state variable. I also use endogenous grid method (EGM) by \cite{carroll2006method}.See Appendix for the detailed solution algorithm.}

The solution to the problem above is the age-specific optimal consumption policies  $c_\tau^*(u_{i,\tau},m_{i,\tau},p_{i,\tau})$ and saving policies $a_\tau^*(u_{i,\tau},m_{i,\tau},p_{i,\tau})$ both as a function of all state variables. 

\subsubsection{Technology}

The economy has a standard CRS technology that turns the capital and supplied efficient units of labor into aggregate output. 

\begin{equation}
    Y = Z K^{\alpha} N^{1-\alpha}
\end{equation}

The capital depreciates at a rate of $\delta$ each period. 

The factors of input markets are fully competitive. Euler Theorem implies that the output either becomes labor income or capital income. 

\subsubsection{Government}

Government runs a balanced budget in each period. Therefore, outlays from unemployment insurances are financed by the income tax that is levied on both labor income and unemployment benefit. Given a replacement ratio $\zeta$, and the proportion of employed population, the corresponding tax rate $\lambda$ can be easily pinned down based on the equation below. \footnote{This convenient result crucially depends on the assumption that unemployment insurance benefit is paid proportionally to permanent income.}

\begin{equation}
\label{Eq:gov}
    \lambda \left[ 1-\Pi^\mho  + \zeta \Pi^\mho \right]  = \zeta \Pi^\mho   
\end{equation}

\subsubsection{Demographics}

For simplicity, we assume there is no population growth. With a deterministic life-cycle profile of survival probabilities, there exists a stable age distribution $\{\mu_\tau \}_{\mu=1,2,..L}$ such that $\mu_{\tau+1} = (1-D)\mu_{\tau}$ and $\sum^{L}_{\tau=1}\mu_{\tau} = 1$. The former condition reflects the probability of survivals at each age and the latter is a normalization that guarantees the fraction of all age groups sum up to 1.\footnote{In a more general setting with a constant population growth rate $n$ and age-specific survival probability $1-D_\tau$, the condition becomes $\mu_{\tau+1} = \frac{(1-D_{\tau+1})}{1+n}\mu_{\tau}\quad \forall \tau = 1, 2...L$, as discussed in \cite{rios1996life} and \cite{huggett1996wealth}.}

\subsubsection{Stationary equilibrium}

Denote $x= \{m,p,\nu \} \in X$  as the idiosyncratic state of individuals. At any point in time, agents in the economy differ in age $\tau$ and their idiosyncratic state $x$. The former is given by $\{\mu_\tau \}_{\mu=1,2,..L}$. For the latter, using $\psi_\tau(B)$ to represent the fraction of agents at age $\tau$ whose individual states lie in $B$ as a proportion of all age $\tau$ agents. ($B$ is essentially a subset of Borel $\sigma$-algebra on state space $X$.) The distribution of age $\tau=1$ agents depend on the initial condition of labor income outcomes and the size of accidental bequest. For any other age $\tau=2...L$, the distribution $\phi_\tau(B)$ evolves as the following.

\begin{equation}
\label{Eq:DistDyn}
\psi_{\tau}(B)=\int_{x \in X} P(x, \tau-1, B) \mathrm{d} \psi_{\tau-1} \quad \text { for all } \quad B \in B(X)
\end{equation}

where $P(x,\tau-1,B)$ is the probability for an agent to transit to $B$ in the next period, conditional on the individual state $x$ at age $\tau-1$. It depends on the optimal consumption policy $c^*(x,\tau)$ at age $\tau$ and the exogenous transition probabilities of income shocks. 

In the absence of the aggregate risk, I focus on the stationary equilibrium of the economy (StE) which consists of consumption and saving policies $c(x,\tau), a(x, \tau)$, constant production factor prices, including real interest rate $R$ and the wage $W$, the initial wealth of newborn $b_1$, unemployment benefit $\zeta$, tax rate $\lambda$ and the time-invariant distribution $\left(\psi_{1}, \psi_{2}, \ldots, \psi_{L}\right)$ such that

1. Consumption and saving policies are optimal given the real interest rate $R$, wage $W$, the tax rate $\lambda$.

$$c(x,\tau)= c^*(x,\tau)$$
$$a(x,\tau)= a^*(x,\tau)$$

2. Distributions $\left(\psi_{1}, \psi_{2}, \ldots, \psi_{L}\right)$ are consistent with optimizing behaviors of household, as described in Equation \ref{Eq:DistDyn}.

3. The factor markets are clearing. 

\begin{equation}
\begin{split}
   & \sum_{\tau} \mu_{\tau} \int_{X}a(x, \tau) \mathrm{d} \psi_{\tau}=K \\
& \sum^{T-1}_{\tau=0} \mu_{\tau} \Pi^E_\tau= N
\end{split}
\end{equation}

4. Firm optimization under competitive factor markets.

$$W = Z(1-\alpha) (K/N)^\alpha $$
$$R = 1+Z\alpha (K/N)^{\alpha-1} - \delta$$


5. Initial bank balance equal to accidental bequests. 

$$b_1=\sum_{\tau} \mu_{\tau}D \int_{x \in X} a(x, \tau)R \mathrm{d} \psi_{\tau}$$
 
6. Government budget is balanced as described in Equation \ref{Eq:gov}.

The economy may potentially arrive at different stationary equilibrium depending on the specific assumptions about objective or subjective models under the configurations. 



\subsection{Subjective risk perceptions}

In the benchmark model, I maintain the FIRE assumption that the agents perfectly know the underlying parameters of income risks $\Gamma =\{\sigma^2_\psi,\sigma^2_\theta,\mho,E\}$ as assumed by the modelers and behave optimally accordingly. 

But here, I relax the FIRE assumption by separately treating the ``true'' underlying risk parameters $\Gamma$ and the risk perceptions held by the agents. The latter is denoted as $\tilde \Gamma_i$. This extension is meant to capture the four empirical patterns documented in the previous sections. 

\begin{enumerate}
    \item Underestimation of the earning risks (compared to what is assumed to be the truth in the model) 
    \item Heterogeneity in risk perceptions 
    \item Extrapolation of recent experiences
    \item State-dependence of risk perceptions 
\end{enumerate}

The possible approaches of capturing these perceptual patterns are by no means unique. I adopt one simple framework that does not require explicitly specified mechanisms of perception formation but sufficient to reflect these the patterns revealed from the survey data.

Assume that each agent $i$ in the economy cannot directly observe the underlying risk parameters $\Gamma$, but instead make his/her best choices based on a subjective risk perceptions  $\tilde \Gamma_{i,\tau}$, which swing between two states: $\tilde \Gamma_l$ (low risk) and $\tilde \Gamma_h$ (high risk). The transition between the two states is governed by a Markov process with a transition matrix $\Omega$. In the calibration of the model in latter sections, these subjective parameters can be estimated from survey data relied upon auxiliary assumptions. 

Such an assumption automatically allows for heterogeneity in risk perceptions across different agents at any point of the time. All individuals are distributed between low and high risk-perception states. In one of the extensions, I also admit ex-ante heterogeneity, namely permanent differences in risk perceptions due to individual fixed effects.  

The transition probability between low-risk and high-risk perception states can be also configured so that the average risk perception is lower than the true level of the risk. If we let the transition matrix $\Omega$ to be dependent on individual unemployment status $\nu_{i,\tau}$, or macroeconomic conditions, we can also easily accommodate the possibility of experience extrapolation and state-dependence feature of risk perceptions. 

Under the assumption of subjective perception, the subjective state of the risk perceptions $\tilde \Gamma$ becomes an additional state variable entering the Bellman equation of the consumer's problem, restated in below.

\begin{equation}
\begin{split}
\tilde V_{\tau}(\tilde \Gamma_\tau, \nu_\tau, m_\tau, p_\tau) = \underset{\{c_\tau\}}{\textrm{max}} \quad u(c_\tau) + (1-D)\beta \mathbb{E}_{\tau}\left[\tilde V_{\tau+1}(\tilde \Gamma_{\tau+1}, \nu_\tau,m_{\tau+1}, p_{\tau+1})\right] 
\end{split}
\end{equation}

Notice here that I assume that the agents recognize the transition between two subjective perception states and take it into account when making the best choices. This assumption guarantees time-consistency and provides additional discipline to the model assumption.  

The consumer's solution to the problem above is the age-specific consumption policy $\tilde c_\tau^*(\tilde \Gamma_\tau,u_\tau, m_\tau,p_\tau)$ that is also a function of subject risk perception state $\tilde \Gamma$.

The distinction between objective and subject risk perception marks the single most important deviation of this paper from existing incomplete-market macro papers. \footnote{For instance, \cite{bewley1976permanent}, \cite{huggett1993risk}, \cite{aiyagari1994uninsured}, \cite{krusell1998income},  \cite{krueger2016macroeconomics},  \cite{carroll2017distribution}.} There is a long tradition of explicitly incorporating various kinds of heterogeneity in addition to uninsured idiosyncratic income shocks in these kinds of models to achieve better match with observed cross-sectional wealth inequality. One of the most notable assumptions used in the literature is the heterogeneity in time preferences (\cite{krusell1998income}, \cite{carroll2017distribution}, \cite{krueger2016macroeconomics}). My modeling approach shares the spirit with and are not mutually exclusive to these existing assumptions on preferential heterogeneity. But, to some extent, perceptual heterogeneity is more preferable as such patterns are directly observed from the survey data, as I show in the previous part of the paper.  


A more fundamental justification for such a deviation from the full information rational expectation assumption is that risk parameters $\Gamma$ are barely observable objects to agents. This is so no matter if they are exogenously assumed by economists or endogenously determined in the equilibrium of the model. \footnote{So far, the majority workhorse incomplete market macro literature has not incorporated any endogenous mechanisms that determine the level of income risks. The emerging literature that incorporates labor market search/match frictions in these models have relied upon simplifying assumptions to get tractability. See, for instance, \cite{mckay2017time,acharya2020understanding,ravn2021macroeconomic},  with the only exception being \cite{ravn2017job}.} Therefore, the conventional argument in favor of rational expectation assumption, namely equilibrium outcome drives the agents' perceptions to converge to the ``truth'', does not apply here. 

Incorporating subjective risk perceptions also alters aggregate dynamics of the distributions as described in Equation \ref{Eq:DistDyn}, as restated below. 

\begin{equation}
\label{Eq:DistDynSub}
\tilde \psi_{\tau-1}(\tilde B)=\int_{\tilde x \in \tilde X} \tilde P(\tilde x, \tau-1, \tilde B) \mathrm{d} \tilde \psi_{\tau-1} \quad \text { for all } \quad \tilde B \in \tilde B(X)
\end{equation}

The state variable $\tilde x$ includes subjective state $\tilde \Gamma$ in addition to those contained in $x$. The transition probabilities $\tilde P$ now depend on the optimal consumption policies $c^*(\tilde x)$ as a function of belief state $\tilde \Gamma$, as well as the exogenous transition probabilities of the true stochastic income process $\Gamma$.  

Then the new StE under subjective risk perceptions can be defined accordingly. 

\begin{comment}
\subsection{Objective versus subjective risk profiles}

%%%%%%%%%%%%%%%%%%%%%%%%%%%%%%%
%% Need to rewrite 
%%%%%%%%%%%%%%%%%%%%%%%%%%%%%%%

%% There is an obvious level effect.  Lower risk perceptions lead to less precautionary savings,  therefore fewer assets holdings. There are two channels via which subjective risk perceptions affect wealth inequality. The first channel is straightforward. The subjective model allows for heterogeneity in risk perceptions. This introduces additional heterogeneity in precautionary saving motives, therefore ex-post savings. The second channel is indirect via lower risk perceptions of all the agents. Lower risk perceptions lead to less optimally chosen self-insurance via savings, therefore, leading to more ex-post wealth inequality compared to a model where perceptions and the truths coincide.

The key implications of the model lie in the differences in consumption functions of the agents in the economy. Therefore, this section devotes to presenting a comparison between the consumption policies under different model assumptions. 

First, I consider the benchmark model, named objective model, under standard parameterization of income risk parameters $\Gamma$ and assumes the unemployment risk to be entirely transitory. 

The first version of the subjective model allows the transition between low and high risk subjective states, under which the average transitory and permanent earning risk is lower than the standard configurations but with the same objective unemployment risk. Call it subjective state-dependent volatility (SV) model. 

The second version of the model further allows the risk perceptions to be dependent upon the idiosyncratic employment status, such that unemployed agents perceive both permanent and transitory risks to be higher than the employed workers do. Call this subjective extrapolation model. 

Figure \ref{fig:comparison1} plots the consumption policies under objective and subjective risk profiles, where agents swing between low and high-risk perceptions. For illustration purpose, I assume the perception state is entirely transitory, which explains why the consumption policy do not differ between low-risk and high-risk perceptions. More generally,  persistence in risk perceptions would induce differences between the two. Also, I configure the model such that the average permanent and transitory risks are exactly equal to the objective model. To put it differently, the subjective model can be seen as a mean-preserving spread of the risk profiles. Whether the precautionary saving motive is intensified in the subjective model or not depends on the level of wealth. For low level of wealth, consumption of consumption is lower than the objective model, as the consumer takes into account the possibility that the risk perceptions can be higher.

%% need to rewrite this .../

\begin{center}
[FIGURE \ref{fig:comparison1} HERE]
\end{center}

Figure \ref{fig:comparison2} further compares the objective model under transitory unemployment risks versus the subjective model in which employment status leads to additional extrapolations in risk perceptions. Specifically, the latter model lets an unemployed worker be with the high-risk perceptions while the employed to be with the low-risk perceptions. This essentially makes the transition matrix $\Omega$ exactly equal to $\Pi$. I also configure risk parameters specific to each state such that on average it preserves the same size of permanent and transitory risks compared to the objective model. This guarantees the difference between the two model is not induced by change in the average degree of precautionary saving motives. 

\begin{center}
[FIGURE \ref{fig:comparison2} HERE]
\end{center}

The most obvious pattern seen in the figure is that within the objective model, unemployed workers have less consumption than employed workers. In addition, between the objective and subjective-extrapolation model, an employed worker with lower risk perceptions actually consume less than the employed worker in the objective model. 
\end{comment}

\subsection{Calibration}


\subsubsection{Estimation of subjective risk profiles}

The parameters to be estimated from the panel data of risk perceptions from $SCE$ are the state-dependent risk profile $\tilde \Gamma_l=\{\tilde \sigma^l_\psi,\tilde \sigma^l_\theta, \tilde{\mho^l}, \tilde{E^l}\}$, $\tilde \Gamma_h =\{\tilde \sigma^h_\psi,\tilde \sigma^h_\theta, \tilde{\mho^h}, \tilde{E^h}\}$ and $\Omega$, the transition matrix between the two states.

Denote the reported risk perception of the individual $i$ at time $t$ in the survey by $\tilde \Gamma^s_{i,t}$. It consists of the underlying risk perceptions relevant to individual decisions, or the model counterpart $\tilde \Gamma_{i,t}$, and an individual-specific, time-specific and an $i.i.d$ shock to the survey responses, respectively. The realization of $\tilde \Gamma_{i,t}$ depends on a hidden state $J_{i,t}$ which is non-observable to economists working with the survey data. It takes value of $1$ if the individual $i$ is at a high-risk-perception state $\tilde \Gamma_{i,t}=\tilde \Gamma_h$ and zero if at low-risk-perceptions $\tilde \Gamma_{i,t} = \tilde \Gamma_l$. The $i.i.d$ shock $\epsilon_{i,t}$ is assumed to follow a mean-zero normal distribution with variance $\sigma^2_\epsilon$.  

\begin{equation*}
	\begin{split}
&	\underbrace{\tilde \Gamma^s_{i,t}}_{\text{reported PR}} = \underbrace{\tilde \Gamma_l + \mathbbm{1}(\overbrace{J_{i,t}}^{\text{Hidden state}}= 1)( \tilde \Gamma_h -\tilde \Gamma_l)}_{\tilde \Gamma_{i,t}} +\xi_{t}+\eta_{i}+ \epsilon_{i,t}\\
& \text{Prob}(J_{i,t+1}|J_{i,t}) = \Omega
\end{split}
\end{equation*}

Notice that the individuals do not separately report their perceived risks for the permanent and transitory shocks, but instead the overall expected income volatility. Therefore, I make an auxiliary assumption that the agent adopts a constant ratio of decomposition between permanent and transitory risks, $\kappa =\frac{\tilde \sigma_{i,t,\psi}}{\tilde \sigma_{i,t,\theta}}$, the value of $\kappa$ is externally estimated from the realized income data. 

In addition, since the surveyed risk perceptions is at the monthly frequency, I estimate the underlying risk parameters for monthly shocks. \footnote{
$\tilde {\var}_{i,t}= (12 \tilde \sigma^2_{i,t,\psi} + 1/12 \tilde \sigma^2_{i,t,\theta})exp^{\xi_{t}}exp^{\eta_{i}}exp^{\epsilon_{i,t}} \rightarrow
\log \tilde {\var}_{i,t}= \log (12 \tilde \sigma^2_{i,t,\psi} + 1/12 \tilde \sigma^2_{i,t,\theta})+\xi_{t}+\eta_{i}+\epsilon_{i,t} \rightarrow
\log(\tilde {\var}_{i,t})= \log [(12+\frac{1}{12\kappa^2})\tilde \sigma^2_{i,t,\psi}] + \xi_{t}+\eta_{i}+ \epsilon_{i,t}$.}

For each individual $i$, we observe at most 12 observations of their perceived income volatility of the earning growth next year $\tilde {var}_{i,t}$ from $t$ to $t+12$ and their job-separation and job-finding expectations, respectively. The panel structure allows the individual fixed effect $\eta_i$ and time-fixed effect $\xi_t$ to be easily identified.  

Then the parameters can be estimated with a modified 2-regime Markov switching model a la \cite{hamilton1989new} using the maximum-log-likelihood (MLE). (See the detailed implementation in Appendix \ref{appendix:markov}). Table \ref{tab:PRMarkovEst}  reports the baseline estimates of the parameters associated with the 2-state Markov model of subjective perceptions. All parameters are converted from monthly into yearly counterparts to be consistent with the model frequency. 

The estimates of subjective profile confirms the key finding we have detailed in the previous section. The estimated staying probabilities at low and high risk perceptions, $q$ and $p$, are around $0.9$, indicating a high degree of persistence in individual risk perceptions. Given these estimated transition probabilities, earning risk perceptions are on average lower than the objective level assumed in the literature.  


\begin{center}
[TABLE \ref{tab:PRMarkovEst} HERE]
\end{center}

\subsubsection{Other parameters}


\textbf{Life-cycle} The model is set at yearly frequency. The working age spans from 25 years old to 65 years old ($T=40$) and the agent dies with certainty at age of 85 $(L=60)$. The constant death probability before the terminal age is set to be $D=0.625\%$. 


As to the deterministic permanent income profile over the life-cycle, $G_\tau$, I draw on an age polynomial regressions of the earning growth from SIPP for workers aged between 25-65, controlling
for other observable demographic variables such as education, gender, occupation, and time fixed effects, etc. This produces very similar estimation results to that in \cite{gourinchas2002consumption}, \cite{cagetti2003wealth} and \cite{kaplan2014model}. The estimated income profile is plotted in Appendix \ref{fig:life-cycle-determinstic}. For the retirement phase, I assume a one-time drop of $20\%$ in permanent income in the age of $66$, i.e. $G_{41}=0.8$, and then the permanent income stays flat till death. This produces an average expected growth rate of permanent income over the entire life-cycle exactly equal to one. This serves as a normalization. Note that although alternative assumptions, such as a more smooth decline of income after retirement, do change the wealth distribution across generations among the retired, they do not change the consumption/saving decisions as such a profile is entirely deterministic. 

\textbf{Initial conditions} Assumptions about cross-sectional distribution of the initial permanent productivity and liquid asset holdings matter for the subsequent wealthy inequality. I set the standard deviation of the log-normally distributed initial permanent individual productivity $p_{i,\tau}$ to be $0.6$ to match the earning heterogeneity in ``usual income'' (approximated permanent income) at age 25 from the SCF. Initial liquid assets holdings at $\tau=0$ is assumed to have a cross-sectional standard-deviation of $0.50$. %% need some reference on this 

\textbf{Income risks} Given the critical importance of the income risks assumption in my model, In addition to my estimates from SIPP, as reported in Table \ref{risk_compare}, I thoroughly survey the parameters used in the existing incomplete market macro literature, as summarized in Table \ref{tab:risks_literature} in the Appendix. For comparison, I convert all risks into the annual frequency (although the model is set quarterly). Whenever group-specific risks are assumed, i.e. depending on the education and age, I summarize it as a range. Also, for those models which assume a persistent instead of permanent income risk component, I treat their assumed size of the persistent risks as a lower bound for the permanent risk. (One can think of the permanent income shock as a limiting case of AR(1) shock, with the persistence parameter infinitely close to 1. The effective income risks increase with the persistence of the shock.)  For models with income risks dependent on aggregate business cycles a la \cite{krusell1998income}, I compute the steady-state size of idiosyncratic risks using the transition probabilities of the aggregate economy used in the paper.  

Despite the disagreements in these estimates or model inputs, the earning risks used in these models are constantly larger than those perceived reported in the survey. Meanwhile, the perceived risk of unemployment is higher in the survey than in these models and the perceived employment probability is lower in the survey than in these models. I use the median values of each parameter in the literature as the objective income risks profile $\Gamma$. In particular, in our baseline calculation, I set $\sigma_\psi = 0.15$ and $\sigma_\theta = 0.10$. The yearly probability of staying on unemployment is $\mho = 0.18$ and that of staying employed $E = 0.96$, as used in \cite{krueger2016macroeconomics}. 

\textbf{Technology} % steady state output to capital ratio % steady state effective interest rate 
The annual depreciation rate is set to be $\delta =2.5\%$. The capital share takes a standard value of $\alpha = 0.36$, for the U.S. economy. Without aggregate shocks, $Z$ is simply a normalizer. Therefore, I set its value such that the aggregate wage rate $W$ is equal to one under a capital/output ratio $K/Y= 3$ at the steady-state level of employment in the model. 

\textbf{Government policies} Unemployment insurance replacement ratio is set to be $\mu=0.15$, as the same as \cite{krueger2016macroeconomics}.  
The pension income relative to the permanent income is assumed to be $\mathbb{S}=60\%$. This, plus the $ 20\%$ drop in permanent income, gives an effective deterministic income drop of $48\%$ from the working-age to retirement, which corresponds to an empirical replacement ratio estimated for the U.S. economy. The corresponding tax rates financing the unemployment insurance and social security is determined in the equilibrium within the model. 

\textbf{Preference} The discount factor is set to be $\beta=0.98$, the average value estimated in the models with heterogeneous time preferences such as \cite{carroll2017distribution}, \cite{krueger2016macroeconomics}. The coefficient of relative risk aversion $\rho=2.0$, which is common in this literature. 

Table \ref{tab:calibration} summarizes the parameters used in this calibration. 

\begin{center}
[FIGURE \ref{tab:calibration} HERE]
\end{center}