\begin{titlepage}
 \title{Perceived Income Risks}
 
 \author{Tao Wang \thanks{Johns Hopkins University, twang80@jhu.edu. I thank Chris Carroll, Jonathan Wright, Robert Moffitt, Yujung Hwang, Francesco Bianchi, Edmund Crawley, Johannes Stroebel, Corina Boar, Yueran Ma, and participants of the behavioral economics conference at Yale SOM for the useful comments. Also, I am thankfull to William Du's help with the development of the Python programs solving the model of the paper.}}

\date{\today \\(Preliminary Draft) \\\href{https://github.com/TaoWangEcon/TaoWangEcon.github.io/blob/master/papers/PerceivedIncomeRisk.pdf}{[Most Recent Draft]}}
	\maketitle
	\begin{abstract}
	\begin{singlespace}
		\noindent Workhorse incomplete-market macro models typically assume that agents have a perfect understanding of the size and nature of income risks that econometricians estimate from past income data. This paper examines if risk perceptions from a representative density survey align with these assumptions. I found that people have reasonable clues about income risks, in that the differences in risk perceptions can be partly explained by between-group differences in income volatility. Perceived earning risks are always lower than the standard estimates based on realized income volatility, suggesting the role of superior information. At the same time, there remains a large degree of heterogeneity. There is robust evidence for state dependence and past dependence. Risk perceptions countercyclically react to recent realizations and negatively correlated with the experiences of macro labor market outcomes. People also extrapolate their own recent experiences of earning volatility and unemployment when forming risk perceptions. These features in risk perceptions have three macroeconomic consequences. First, lower perceived risks on average helps account for the concentration of low liquid wealth holding among the population. Second, the heterogeneity in risk perceptions leads to additional heterogeneity in saving behavior and marginal propensity to consume (MPC). Third, state-dependent income risk perceptions induce additional precautionary saving motives, and depending on its cyclicality, could further amplify or dampen the business cycle fluctuations of aggregate consumption. My ongoing work explores the quantitative importance of these predictions in a general-equilibrium incomplete market model. 
	\end{singlespace}

		\noindent \textbf{Keywords: Income risks, Incomplete market, Perception, Precautionary saving } \\
		\noindent \textbf{JEL Codes: D14, E21, E71, G51} 

	\end{abstract}

\end{titlepage}
