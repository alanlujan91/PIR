\begin{titlepage}
 \title{Perceived Income Risks}
 
 \author{Tao Wang \thanks{Johns Hopkins University, twang80@jhu.edu. GitHub page: \url{http://github.com/iworld1991/PIR}. I thank Chris Carroll, Jonathan Wright, Robert Moffitt, Yujung Hwang, Francesco Bianchi, Edmund Crawley, Mateo Velasquez-Giraldo, Johannes Stroebel, Corina Boar, Yueran Ma, Xincheng Qiu, and seminar participants of the PhD conference at Yale SOM for the useful comments. Also, I am thankfull to William Du's contribution to the development of computational methods in this paper.}}

\date{\today \\(Preliminary Draft) \\\href{https://github.com/iworld1991/PIR/blob/master/PIR.pdf}{[Most Recent Draft]}}
	\maketitle
	\begin{abstract}
	\begin{singlespace}
		\noindent The state-of-art incomplete-market macro models featuring uninsured idiosyncratic income risks typically use estimated risks from cross-sectional income realizations. But this practice could run into the problem of unobserved heterogeneity, and cannot perfectly approximate the income shocks from the point of view of the agents. This paper calibrates the income risks in a standard OLG/incomplete-market model using a large-scale representative exceptional survey, which directly elicits density forecasts of individuals' wage growth. It shows that incorporating a number of salient facts of risks as revealed by reported perceptions in the survey, helps account for the low liquidity-asset holding of a large fraction of agents in the U.S. economy, i.e. hands-to-mouth consumers, and the wealth inequality seen in the data. I also extend the model to allow for possible behavioral bias in perceiving risks by agents and explore its macroeconomic consequences. This extension also serves as an experiment model that breaks down the effects of idiosyncratic income risks on wealth inequality into two channels: ex-ante saving behaviors and the ex-post realized income inequality.
		
		% Workhorse incomplete-market macro models typically assume that agents have a perfect understanding of the size and nature of income risks that econometricians estimate from past income data. This paper examines if risk perceptions from a representative density survey align with these assumptions. I found that people have reasonable clues about income risks, in that the differences in risk perceptions can be partly explained by between-group differences in income volatility. Perceived earning risks are always lower than the standard estimates based on realized income volatility, possibly due to unobserved heterogeneity or overconfidence. At the same time, there remains a large degree of heterogeneity. There is robust evidence for state dependence and past dependence. Risk perceptions countercyclically react to recent realizations and negatively correlated with the experiences of macro labor market outcomes. People also extrapolate their own recent experiences of earning volatility and unemployment when forming risk perceptions. These features in risk perceptions have three macroeconomic consequences. First, lower perceived risks on average helps account for the concentration of low liquid wealth holding among the population. Second, the heterogeneity in risk perceptions leads to additional heterogeneity in saving behavior and marginal propensity to consume (MPC). Third, state-dependent income risk perceptions induce additional precautionary saving motives, and depending on its cyclicality, could further amplify or dampen the business cycle fluctuations of aggregate consumption. My ongoing work explores the quantitative importance of these predictions in a general-equilibrium incomplete market model. 
		
	\end{singlespace}
		\noindent \textbf{Keywords: Income risks, Incomplete market, Perception, Precautionary saving } \\
		\noindent \textbf{JEL Codes: D14, E21, E71, G51} 

	\end{abstract}

\end{titlepage}
