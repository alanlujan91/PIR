\hypertarget{unemployment-risk-perceptions}{%
\subsection{Unemployment risk perceptions}\label{unemployment-risk-perceptions}}

The analysis so far only focuses on the wage risks conditional on staying in the same job. But it admittedly only constitutes a lower bound of the labor market risks since major events such as job loss and switching usually result in more significant changes in labor income and affects a household's welfare\footnote{\cite{low2010wage}, \cite{davis2011recessions}.}. Unemployment risks are usually another central input of the incomplete-market macroeconomic models. Similar to the approach with earning risks, the common practice in these models is to model the process of labor market transitions based on externally estimated statistics and assume the agent correctly perceives it within the model. Therefore, by the same token as for the earning risk, this section examines if the survey-reported expectations of job separation align with realized aggregate dynamics revealed from the labor market statistics. Moreover, are perceived unemployment risks extrapolate from recent experiences of aggregate and individual labor market outcomes?


For a fair comparison between perceptions and realizations which are available at different horizons, we cast both probabilities into a continuous-time Poisson rate. Specifically, for the expectation, let the reported probability of separating from the current job in next 12 months be $P_{i,t}(ue_{t+12}|e_t)$, then the corresponding monthly Poisson rate of job-separation $E_{i,t}(s_{t+1})$ is $- log(1-P_{i,t}(u_{t+12}|e_t))/12$\footnote{This follows from the following mathematical fact: for a continuous-time Poisson process with an event rate of $\theta$, the arrival probability over a period of $\Delta t$ units of time is equal to $1-exp^{-\theta \Delta t}$.}. With the realized month-to-month flow rate estimated from CPS $P(ue_{t+1}|e_t)$ , the corresponding realized Poisson rate  $s_{t+1}$ is $-log(1-P(ue_{t+1}|e_t))$. 


Figure \ref{fig:srate_compare} plots the converted job-separation expectations and realizations against each other. A few important patterns emerge. First, the comparison confirms our earlier findings based on earning risks that individual perceptions did track the aggregate realizations relatively well, assuring us that people's self-reported perceptions are not entirely groundless or naive. But on the other hand, there are systematic differences between perceptions and realizations. 

\begin{center}
[FIGURE \ref{fig:srate_compare} HERE]
\end{center}



