
\hypertarget{related-literature}{%
\subsection*{Related literature}\label{related-literature}}

First, this paper closely builds on the literature estimating both cross-sectional and time trends of labor income risks and partial insurance. Early work estimating income risks includes \cite{gottschalk1994growth, carroll1997nature}.  Later, the literature explores time-varying patterns of the income risks. For instance, \cite{meghir2004income} allows for time-varying risks or conditional heteroscedasticity in the traditional permanent-transitory model. \cite{blundell_consumption_2008} uses the same specification to income to estimate partial insurance in conjunction with consumption data. More recently,  \cite{bloom2018great} found the risks have declined in recent decades. Moreover, recent evidence relied upon detailed social security records and larger data samples highlight the asymmetry and cyclical behaviors of idiosyncratic earning/income risks \citep{storesletten2004cyclical, guvenen2014nature,arellano2017earnings,guvenen2019data}. Besides, a separate literature focus on job-separation and unemployment risks\citep{low2010wage,davis2011recessions}. The novelty of this paper lies in the focus on the subjective perceptions of labor risks and how it is correlated with the realized income risks estimated from the income panel. 


Second, it
is related to an old but recently reviving interest in studying
consumption/saving behaviors in models incorporating imperfect
expectations and perceptions. For instance, the closest to the current
paper, \cite{pischke1995individual} explores the implications of the
incomplete information about aggregate/individual income innovations by
modeling agent's learning about inome component as a signal extraction
problem. \cite{wang2004precautionary} extends the framework to
incorporate precautionary saving motives. In a similar spirit,
\cite{carroll_sticky_2018} reconciles the low micro-MPC and high
macro-MPCs by introducing to the model an information rigidity of
households in learning about macro news while being updated about micro
news. \cite{rozsypal_overpersistence_2017} found that households'
expectation of income exhibits an over-persistent bias using both
expected and realized household income from Michigan household survey.
The paper also shows that incorporating such bias affects the aggregate
consumption function by distorting the cross-sectional distributions of
marginal propensity to consume(MPCs) across the population.
\cite{lian2019imperfect} shows that an imperfect perception of wealth
accounts for such phenomenon as excess sensitivity to current income and
higher MPCs out of wealth than current income and so forth. My paper has
a similar flavor to all of these works in that I also explore the
behavioral implications of households' perceptual imperfection. But it
has important two distinctions. First, this paper focuses on higher
moments such as income risks. Second, most of these existing work either
considers inattention of shocks or bias introduced by the model
parameter, none of these explores the possible misperception of the
nature of income shocks.
\footnote{For instance, \cite{pischke1995individual} assumes that agents know perfectly about the variance of permanent and transitory income so that they could filter the two components from observable income changes. This paper instead assumes that that the agents do not observe the two perfectly.}

Third, empirically, this paper also contributes to the literature
studying expectation formation using subjective surveys. There has been a long list of ``irrational expectation'' theories developed in recent decades on how agents deviate from full-information rationality
benchmark, such as sticky expectation, noisy signal extraction, least-square learning, etc. Also, empirical work has been devoted to
testing these theories in a comparable manner (\cite{coibion2012can},
\cite{fuhrer2018intrinsic}). But it is fair to say that thus far,
relatively little work has been done on individual variables such as labor income, which may well be more relevant to individual economic decisions. Therefore, understanding expectation formation of the individual variables, in particular, concerning both mean and higher moments, will provide fruitful insights for macroeconomic modeling assumptions.

Fourth, the paper is indirectly related to the research that advocated
for eliciting probabilistic questions measuring subjective uncertainty
in economic surveys (\cite{manski_measuring_2004},
\cite{delavande2011measuring}, \cite{manski_survey_2018}). Although the
initial suspicion concerning to people's ability in understanding, using
and answering probabilistic questions is understandable,
\cite{bertrand_people_2001} and other works have shown respondents have
the consistent ability and willingness to assign a probability (or
``percent chance'') to future events. \cite{armantier_overview_2017}
have a thorough discussion on designing, experimenting and implementing
the consumer expectation surveys to ensure the quality of the responses.
Broadly speaking, the advocates have argued that going beyond the
revealed preference approach, availability to survey data provides
economists with direct information on agents' expectations and helps
avoids imposing arbitrary assumptions. This insight holds for not only
point forecast but also and even more importantly, for uncertainty,
because for any economic decision made by a risk-averse agent, not only
the expectation but also the perceived risks matter a great deal.

Lastly, the idea of this paper echoes with an old problem in the
consumption insurance literature: `insurance or information'
(\cite{pistaferri_superior_2001},
\cite{kaufmann_disentangling_2009}, \cite{meghir2011earnings}, \cite{kaplan2010much}). In any
empirical tests of consumption insurance or consumption response to
income, there is always a worry that what is interpreted as the shock
has actually already entered the agents' information set or exactly the
opposite. For instance, the notion of excessive sensitivity, namely
households consumption highly responsive to anticipated income shock,
maybe simply because agents have not incorporated the recently realized
shocks that econometricians assume so (\cite{flavin_excess_1988}). Also,
recently, in the New York Fed
\href{https://libertystreeteconomics.newyorkfed.org/2017/11/understanding-permanent-and-temporary-income-shocks.html}{blog},
the authors followed a similar approach to decompose the permanent and
transitory shocks. My paper shares a similar spirit with these studies
in the sense that I try to tackle the identification problem in the same
approach: directly using the expectation data and explicitly controlling
what are truly conditional expectations of the agents making the
decision. This helps economists avoid making assumptions on what is
exactly in the agents' information set. What differentiates my work from
other authors is that I focus on higher moments, i.e.~income risks and
skewness by utilizing the recently available density forecasts of labor
income. Previous work only focuses on the sizes of the realized shocks
and estimates the variance of the shocks using cross-sectional
distribution, while my paper directly studies the individual specific
variance of these shocks perceived by different individuals.

