\hypertarget{experiments}{%
\section{Model implications (preliminary)}\label{experiments}}

\subsection{Baseline model results}

I first examine the wealthy inequality generated from the benchmark objective model with the income risks calibrated following the existing literature such as \cite{krueger2016macroeconomics,carroll2017distribution}. In particular, I use the standard parameterization on permanent and transitory risks at the annual frequency being $\sigma_\psi=0.10$ and $\sigma_\theta =0.15$, and the unemployment risks $U2U=0.18$ and $E2E=0.96$. The upper panel of Figure \ref{fig:StE_dist_objective} reproduces the well-known result\footnote{For a thorough survey on this topic, see \cite{guvenen2011macroeconomics}, \cite{de2015quantitative}, and \cite{kaplan2018microeconomic}. } that a standard incomplete market model, without additional heterogeneity such as that in time discount rates, imply less (either in partial equilibrium or general equilibrium) wealth inequality than that seen in the data. \footnote{I use 2016 vintage of the SCF. }

\begin{center}
[FIGURE \ref{fig:StE_dist_objective} HERE]
\end{center}

The life-cycle profile of wealth is another dimension of the model implications that can be compared to its counterparts seen in the data. The bottom panel of Figure  \ref{fig:StE_dist_objective} plots the hump-shaped average wealth over the life cycle implied by the model against the median net wealth between the age of 25 to 85 from SCF. There are two divergences between the model and the data. First, compared to the data, the model implies a more rapid build-up of buffer-stock wealth before the middle age. This is so because of the precautionary saving motives in the presence of income risks of various kinds in this model. 

The second divergence has to do with the high wealth of the old seen in the data, in comparison with the sharp decline in wealth toward zero in the model, as the latter dictates it is optimal to consume all wealth in the end of the life. Such a divergent pattern is usually accounted for by specifically taking into account the bequest motives in the literature.\footnote{\cite{de2004wealth}.} As the focus of this paper is on labor income risks of the work-age, I choose not to model such a mechanism. 

The difference between the partial equilibrium and general equilibrium in terms of wealth inequality is rather small. Instead, allowing for the endogenous determination of real interest rate in the asset markets clearing as well as the resource constraint imposed by government budget balancing induces a steeper build-up of the buffer stock savings by the young. This is partly due to a higher real return to savings is higher in GE than PE. 

\subsection{Model results with survey-implied risks}

I incorporate the two salient facts as revealed in perceived income risks and explore their implications, respectively. First, an average lower income risk, possibly due to superior information by the agents. Call it the $LPR$ (lower perceived risks) model. Second, the heterogeneity in perceived risks. Call it a $HPR$ (heterogeneous perceived risks) model. I examine their implications separately compared to the benchmark model. 

\subsubsection{LPR}

For LPR calibration, I keep everything the same as baseline calibration above, except for setting the permanent and transitory risks to be smaller based on an upper bound of total perceived risk of $0.04$, i.e. $\sigma_\psi=0.03$ and $\sigma_\theta=0.02$. 

Figure \ref{fig:StE_dist_compare} confirms the two straightforward implications of a smaller size of risks. First, a lower PR induce less precautionary saving motive and reduces buffer-stock savings of all working agents, as indicated by a lower line of the wealth before retirement than the benchmark model. This also means that there will be a leftward shift in wealth distribution of the entire population, therefore increasing the proportion of the agents that are close to zero wealth. 

Second, if it does reflect an objectively lower size of the idiosyncratic risks than the usual estimation in the benchmark model, a lower PR unambiguously leads to \emph{less} wealth inequality than in the benchmark model, as shown in the Lorenz curve in Figure \ref{fig:StE_dist_compare}. This means that the simple LPR extension cannot help explain an additional degree of wealth inequality as seen in the data.

\begin{center}
[FIGURE \ref{fig:StE_dist_compare} HERE]
\end{center}

A lower wealth inequality could be either due to an objectively higher income inequality, which is exogenous to the agents in the model, or partly endogenous responses via consumption/saving decisions. In order to separate the two channels, I solve the model above by separately parameterizing the risks as perceived by the agents (subjective) and the level of risks that determine the realized income distributions (objective). In particular, I set the perceived risks that affect agents' consumption/saving decisions to be the low values as reported in the survey, like in LPR, while keeping the objective risk parameters the same as the baseline model. Call this model SLPR (subjective LPR model). % need better names. 

SLPR can be thought of as a thought experiment I invoked to disentangle the effects via choices and realizations. It can also be a model which accommodates more general possibilities of discrepancies between perceptions and the objective model counterpart,  which deviates from the assumption of rational expectations. Appendix \ref{appendix:subjective_model} discusses in greater detail on how the consumer problem, the dynamics and stationary distribution of the economy change within a model allowing for subjective risk perceptions and the objective risks to be different. 


The difference between SLPR and the baseline model reflects the changes solely attributable to responses in saving behaviors under a lower risk perceptions, and the difference between SLPR and LPR corresponds to income inequality. As shown in Figure \ref{fig:StE_dist_compare}, the income inequality channel contributes to approximately 80\% of the decreases in inequality and the rest is attributable to saving behavior changes. In this class of models, both permanent and transitory income shocks are at most partially insured via self-insurance. This explains why saving behaviors have a smaller effect in driving the distribution than income inequality. 


\subsubsection{HPR}

Compared to the baseline model, HPR model assumes heterogeneous idiosyncratic risks due to stochastic transitions between different levels of risks, as seen from the survey data. This is to model the state-dependence in risk perceptions of individuals, depending on the information set available to agents unobserved by economists. It is also consistent with the observation that individuals in SCE does change their reported risk perceptions from time to time. These changes are possibly due to purely idiosyncratic reasons which economists have no way of knowing, even controlling time individual effects. Such an assumption of stochastic/state-dependent risk assumes that the heterogeneity in risks is not due to ax-ante time-invariant heterogeneity in the level of risks facing everyone. In one of the latter extension, I also allow for ex ante/constant heterogeneity in risks, as estimated from individual fixed effects in PR.  

Stochastic risks results in more wealth inequality compared to the baseline model. This is so simply because different risks induce different precautionary saving motives, therefore, different buffer-stock savings. In addition, the transitions across states with different risks at individual levels also generate more income inequality, separate from the saving behaviors. 

As in LPR, I also consider an intermediate model experiment, allowing for the divergence between the perceived risks and that objective risks that drive income inequality. In particular, I assume that in SHPR, agents choose saving/consumption believing the stochastic nature of income risks while the realized income inequality is determined by homogenous degree of income risks. Call it SHPR (subjective HPR) model. The difference between HPR and SHPR speaks to the response in saving behaviors in response to heterogeneity in risks due to the state-dependence. The difference between baseline and SHPR corresponds to the additional inequality that comes from stochastic income risks.  Alternatively, this model can be thought of as one where agents do not recognize the heterogeneity in income risks. 

\subsection{Extensions}

\subsubsection{Ex-ante heterogeneity in risks}

HPR model incorporates heterogeneity in risks via an assumption of stochastic risks at individual level. But as shown in Section \ref{cross-sectional-heterogeneity} and the estimation from survey in Section \ref{subsubsec:survey_risk}, a large degree of heterogeneity in PR is attributable to individual fixed effects, which might reflect the true ex ante heterogeneity in income risks facing different individuals. Since there are no individual realizations of risks, traditional risk estimation based on cross-sectional income cannot recover the heterogeneity in risks unless grouping individuals by certain group characteristics. Survey-reported PRs, in contrast, allows for direct identification of the cross-sectional heterogeneity across individuals. The advantage of this approach is that modelers can be agnostic about the true reasons for the heterogeneity, and it avoids making arbitrary model specifications as to the heterogeneity in income risks.

Ex-ante heterogeneity in risks in addition to the stochastic transitions unambiguously contributes to more wealth inequality. Again, the effect consists of changes due to saving behaviors via heterogeneous precautionary saving motives, and the changes due to various degree of income risks. 

\subsubsection{Unemployment risks}

The bulk of the results so far has narrowly focused on only calibrating perceived wage risks using the survey. A natural extension is to incorporate the survey-informed unemployment risks, i.e, the heterogeneity and state-dependence in perceived job loss and job finding probabilities, in the same model. Standard incomplete market model with unemployment spells typically parameterize the model with one homogenous pair of $U2U$ and $E2E$ probabilities. But this may mask the unobserved heterogeneity among agents and their true perceived unemployment risks given the information they have about their own idiosyncratic circumstances (\cite{mueller2021expectations}).  

As detailed in Section \ref{subsubsec:survey_risk}, I recover two unobserved states and their transition probabilities by jointly accounting for the changes in reported wage risk, separation and finding expectations. I also further recover the ex-ante differences in risk perceptions using the same approach above. 

The resulting model embodies both ex-ante and ex-post heterogeneity in wage risks as well as unemployment risks. By the same token, I introduce the experiment model allowing agents to make saving decisions with such heterogeneity in both wage risks and unemployment risks to be perceived, but I keep the underlying income inequality to be driven by standard homogenous degree of income inequality. 

\subsubsection{Job-switching wage loss and risks}

Persistent unemployment spells cause income reductions due to the forgone wage that would have been otherwise earned in employment, but it also induces persistent income loss even after reemployment to a new job. It is worth extending the model above to explicitly consider the latter channel. Table \ref{tab:scarring_literature} summarizes a large body of micro empirical literature that has found evidence for such an additional channel via which income risks matter for income and wealth inequality. Most of the estimates summarized focuses on mass layoffs, but it also shed light on the job-displacement costs following purely idiosyncratic unemployment spells followed by job-switching. Some recent job search and match models \footnote{Such as \cite{low2010wage}, \cite{lachowska2020sources}, etc.} rationalizes such a persistent wage loss associated with unemployment spells with a job-ladder model with on-the-job search. Some other literature attributes the wage loss to employee-employer specific human capital. And some papers emphasize the human capital decay specific to the individuals. 

Regardless of the mechanism, I simply assume in the extended model that there is a one-time wage loss associated with each transition from unemployment to employment. I parameterize it using the median estimate in the literature of a $10\%$ wage loss that permanently affects the wage earned at the same job for each employment spell. 

A closely related yet separate effect associated with  unemployment-employment transition is the job-switching wage risk. Each time the worker is reemployed by a new job or new employer, there are possibly good or bad wage rates drawn stochastically specific to this new job. This is separate from the source of income risks modeled so far.  \cite{low2010wage} emphasizes the importance of separately identifying the conditional wage risk and the job-switching risk, using 1993 SIPP data. Their estimation suggests a standard deviation of the latter component of $20\%$, well above the upper bound of the wage risks assumed in all the configurations above. 

Introducing these two channels to the baseline model would undoubtedly increase the income inequality and induce additional precautionary saving motives. It will increase the model-generated wealth inequality.  