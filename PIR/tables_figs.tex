                        \newpage 
   
  \section*{Tables and Figures} 
    
    %%%%%%%%%%%%%%%%%%%%%%%%%%%%%
    % figures 
    %%%%%%%%%%%%%%%%%%%%%%%%%%%%%
    
    
    \begin{figure}[!ht]
    	\caption{Perceived Risks, Wage Volatility and Estimated Wage Risks by Observable Factors}
    	\label{fig:group_compare}
    	\begin{center}
    	%\adjustimage{max size={0.7\linewidth}}{figures/real_log_wage_shk_gr_by_age_edu_gender_compare.png}
    		\adjustimage{max size={0.68\linewidth}}{figures/boxplot_rvar_compare_educ.png} \\
    		\medskip
    	\adjustimage{max size={0.68\linewidth}}{figures/boxplot_rvar_compare_age.png}
    	\end{center}
    	
    	\begin{flushleft}Note: this figure plots perceived risk (from SCE), average estimated wage volatility, approximated wage risk, and permanent risk (from SIPP) of each education-gender (upper panel) or age-gender (bottom panel) group. The volatility is approximated by the within-group cross-sectional standard deviation of log changes in unexplained wage residuals, as defined in Equation \ref{Eq:definition_volatility}. The estimated risk is based on the process specified in Equation \ref{Eq:income_volatility}.\end{flushleft}
    \end{figure}

    \clearpage
    \begin{figure}[!ht]
    	\caption{Perceived Wage Risks by Earning Decile}
    	\label{fig:barplot_byinc}
    	\begin{center}\adjustimage{max size={0.7\linewidth}}{figures/boxplot_rvar_earning.png}\end{center}
    \begin{flushleft}Note: this figure plots average perceived income risks by the decile of annual earning of the same individual.\end{flushleft}
    \end{figure}
    
    
    \clearpage
    
        \begin{figure}[!ht]
    \caption{Dispersion in Unexplained Perceived Income Risks}
    \label{fig:histmoms}
	\begin{center}
    		\adjustimage{max size={0.4\linewidth}{0.3\paperheight}}{figures/joy_incstd.jpg}
    		\adjustimage{max size={0.4\linewidth}{0.3\paperheight}}{figures/joy_rincstd.jpg}
\end{center}
    \begin{flushleft}Note: this figure plots the distributions of residuals of the perceived standard deviation of 1-year-ahead earning growth in nominal (left) and real terms (right) after controlling age, age polynomial, gender, education, type of work arrangement, and time fixed effects. The real risk is the sum of the perceived risk of nominal income and inflation uncertainty.\end{flushleft}
    	% average PR: 2.1% in std; 10/90 IQR: 3.2% in std
    \end{figure}
    
    \clearpage
    
    \begin{figure}[!ht]
    	\caption{Perceived and Realized Risks}
    	\label{fig:ts_compare}
    	\begin{center}
    	\adjustimage{max size={0.5\linewidth}}{figures/real_volatility_compare.png}
    		\vbiskip
    		\adjustimage{max size={0.5\linewidth}}{figures/real_permanent_compare.png}
    	\vbiskip
    	\adjustimage{max size={0.5\linewidth}}{figures/real_transitory_compare.png}
    	\end{center}
    \begin{flushleft}Note: this figure plots median 1-year-ahead perceived income risks in the whole SCE sample against the estimated realized risks, permanent, and transitory risks over the \emph{same} period. Both series are regarding the real wage.The realized risks are first estimated monthly from SIPP and then aggregated into annual frequency. Specifically, the permanent risks are the sum of monthly permanent risks and the annual transitory risks are the simple average over the corresponding 12 months.\end{flushleft}
    \end{figure}
    
  
    \clearpage
    \begin{figure}[!ht]
      \caption{Recent Labor Market Conditions and Perceived Risks}
    \label{fig:ts_he}
    	\begin{center}\adjustimage{max size={\linewidth}}{figures/tsMean3mvrvar_he.jpg}
    \end{center}
    \begin{flushleft}Note: recent labor market outcome is measured by hourly wage growth (YoY). The 3-month moving average is plotted for both series.\end{flushleft}
    \end{figure}
    
     
    \clearpage
    \begin{figure}[!ht]
      \caption{Expected and Realized Job-separation/finding Rate}
    \label{fig:srate_compare}
    	\begin{center}\adjustimage{max size=0.7\linewidth}{figures/separation_rate_1y.png} \\
    	\medskip
    	\adjustimage{max size=0.7\linewidth}{figures/job_finding_rate.png}
    	\end{center}
    \begin{flushleft}Note: realized job separation/finding rates are computed from CPS. Both are expressed as Poisson point rates in continuous time with one month as the unit of time. The 3-month moving average expected rate is plotted.\end{flushleft}
    \end{figure}
    
 \newpage
 


 \newpage
 
 \begin{comment}
 
\begin{figure}[!ht]
	\caption{Consumption functions under objective and subjective risk profiles}
	\label{fig:comparison1}
	\begin{center}
		\adjustimage{max size={0.8\linewidth}{0.5\paperheight}}{figures/comparison1.png}
\end{center}
\begin{flushleft}Note: this figure plots an example of age-specific optimal consumption policies under objective and subjective state-dependent income risk profiles, respectively. The two models use the exact same parameter configurations except for the subjective model having agents to stochatically draw a low or high risk perception. Both the average size of transitory and permanent risks are kept equal to that in the objective model.\end{flushleft}
\end{figure}

\newpage 

\begin{figure}[!ht]
	\caption{Consumption functions under objective and state-dependent risk profiles}
	\label{fig:comparison2}
	\begin{center}
		\adjustimage{max size={0.8\linewidth}{0.5\paperheight}}{figures/comparison2.png}
\end{center}
\begin{flushleft}Note: this figure plots age-specific optimal consumption policies under objective and subjective extrapolation model. The former model includes persistent unemployment risks while the latter maintains the same transition probability of employment status but allows the risk perceptions to be dependent upon the employment status of the individual, i.e. unemployed perceive income risks to be higher than the employed.\end{flushleft}
\end{figure}
\end{comment}


\clearpage


\begin{figure}[!ht]
	\caption{Wealth inequality in Partial and General Equilibrium: baseline model}
	\label{fig:StE_dist_objective}
	\begin{center}
		\adjustimage{max size={0.3\linewidth}{0.4\paperheight}}{figures/lorenz_curve_a_eq.png}
	\adjustimage{max size={0.6\linewidth}{0.4\paperheight}}{figures/life_cycle_a_test.png} \\
		\adjustimage{max size={0.3\linewidth}{0.4\paperheight}}{figures/lorenz_a_test.png} 
	\adjustimage{max size={0.6\linewidth}{0.4\paperheight}}{figures/life_cycle_a_eq.png}
		
	
\end{center}
\begin{flushleft}Note: the panel shows, under the baseline objective model, the Lorenz curve of households wealth (left) and the model-generated life-cycle profile of log average wealth compared to that seen in the 2016 vintage of Survey of Consumer Finance (SCF) (right) in the partial equilibrium (upper panel) and in general equilibrium (bottom panel). \end{flushleft}
\end{figure}


\begin{figure}[!ht]
	\caption{Wealth inequality in Partial and General Equilibrium: under Different Model Assumptions}
	\label{fig:StE_dist_compare}
	\begin{center}
		\adjustimage{max size={0.3\linewidth}{0.4\paperheight}}{figures/lorenz_a_compare_pe.png}
	\adjustimage{max size={0.6\linewidth}{0.4\paperheight}}{figures/life_cycle_a_compare_pe.png} \\
		\adjustimage{max size={0.3\linewidth}{0.4\paperheight}}{figures/lorenz_curve_a_compare_ge.png}
	\adjustimage{max size={0.6\linewidth}{0.4\paperheight}}{figures/life_cycle_a_compare_ge.png} \\
	
\end{center}
\begin{flushleft}Note: the panel shows, under different assumptions, the Lorenz curve of households wealth (left) and the model-generated life-cycle profile of log average wealth compared to that seen in the 2016 vintage of Survey of Consumer Finance (SCF) (right) in the partial equilibrium (upper panel) and in general equilibrium (bottom panel). \end{flushleft}
\end{figure}


%%%%%%%%%%%%%%%%%%%%%%%%%%%%%%%%%%%%%%% % tables   
%%%%%%%%%%%%%%%%%%%%%%%%%%%%%%%%%%%%%
\clearpage


\begin{table}[p]
\centering
\begin{adjustbox}{width=\textwidth}
\begin{threeparttable}
\caption{Covariants of Perceived Wage Risks}
\label{micro_reg}\begin{tabular}{lllllll}
\toprule
{} & incvar I & incvar II & incvar III & incvar IIII & incvar IIIII & incvar IIIIII \\
                    &          &           &            &             &              &               \\
\midrule
IdExpVol            &  4.58*** &   2.23*** &    2.69*** &     2.75*** &      2.95*** &       2.94*** \\
                    &   (0.33) &    (0.36) &     (0.39) &      (0.39) &       (0.38) &        (0.39) \\
AgExpVol            &     0.04 &   0.28*** &    0.34*** &     0.32*** &      0.18*** &       0.20*** \\
                    &   (0.04) &    (0.04) &     (0.05) &      (0.05) &       (0.05) &        (0.05) \\
AgExpUE             &  0.14*** &   0.08*** &     0.05** &       0.05* &        0.04* &        0.05** \\
                    &   (0.02) &    (0.02) &     (0.02) &      (0.02) &       (0.02) &        (0.02) \\
age                 &          &  -0.02*** &   -0.02*** &    -0.02*** &     -0.02*** &      -0.02*** \\
                    &          &    (0.00) &     (0.00) &      (0.00) &       (0.00) &        (0.00) \\
gender=male         &          &           &   -0.36*** &    -0.35*** &     -0.32*** &      -0.30*** \\
                    &          &           &     (0.02) &      (0.02) &       (0.02) &        (0.02) \\
nlit\_gr=low nlit    &          &           &    0.09*** &     0.09*** &      0.10*** &       0.09*** \\
                    &          &           &     (0.02) &      (0.02) &       (0.02) &        (0.02) \\
parttime=yes        &          &           &            &             &        -0.01 &         -0.02 \\
                    &          &           &            &             &       (0.02) &        (0.02) \\
selfemp=yes         &          &           &            &             &      1.25*** &      -0.00*** \\
                    &          &           &            &             &       (0.03) &        (0.00) \\
UEprobAgg           &          &           &            &             &              &       0.02*** \\
                    &          &           &            &             &              &        (0.00) \\
UEprobInd           &          &           &            &             &              &       0.02*** \\
                    &          &           &            &             &              &        (0.00) \\
HHinc\_gr=low income &          &           &            &             &      0.16*** &       0.16*** \\
                    &          &           &            &             &       (0.02) &        (0.02) \\
educ\_gr=high school &          &           &            &    -0.10*** &     -0.13*** &      -0.09*** \\
                    &          &           &            &      (0.02) &       (0.02) &        (0.02) \\
educ\_gr=hs dropout  &          &           &            &        0.08 &         0.11 &       0.29*** \\
                    &          &           &            &      (0.11) &       (0.11) &        (0.11) \\
N                   &    41422 &     41422 &      34833 &       34833 &        33480 &         29687 \\
R2                  &     0.01 &      0.02 &       0.04 &        0.04 &         0.11 &          0.06 \\
\bottomrule
\end{tabular}
	\begin{flushleft}
\item Standard errors are clustered by household. *** p$<$0.001, ** p$<$0.01 and * p$<$0.05. 
\item This table reports results associated a regression of logged perceived income risks (incvar) on logged idiosyncratic($\text{IdExpVol}$), aggregate experienced volatility($\text{AgExpVol}$), experienced unemployment rate (AgExpUE), and a list of household specific variables such as age, income, education, gender, job type and other economic expectations.
\end{flushleft}
\end{threeparttable}
\end{adjustbox}
\end{table}
\clearpage

\begin{table}[p]
\centering
\begin{adjustbox}{width={0.9\textwidth}}
\begin{threeparttable}
\caption{Perceived Income Risks and Household Spending Plan}
\label{spending_reg}
\begin{tabular}{lllllll}
	\hline 
	& (1)      & (2)      & (3)      & (4)      & (5)      & (6)      \\
		\hline 
	perceived earning risk           & 8.394*** & 8.399*** & 3.642*** & 3.243*** &          &          \\
	& (1.175)  & (1.176)  & (0.533)  & (0.537)  &          &          \\
	&          &          &          &          &          &          \\
	perceived earning risk (nominal) &          &          &          &          & 3.656*** &          \\
	&          &          &          &          & (0.990)  &          \\
	&          &          &          &          &          &          \\
	perceived ue risk                &          &          &          &          &          & 0.353*** \\
	&          &          &          &          &          & (0.0553) \\
		\hline 
	R-squared                        & 0.0010 & 0.00282  & 0.928    & 0.928    & 0.941    & 0.633    \\
	Sample Size                      & 53178    & 53178    & 53178    & 53178    & 54584    & 6269     \\
	Time FE                          & No       & Yes      & No       & Yes      & Yes      & No       \\
	Individual FE                    & Yes       & No       & Yes      & Yes      & Yes      & Yes     \\
		\hline 
\end{tabular}
	\begin{flushleft}
		{\footnotesize\item Standard errors are clustered by household. *** p$<$0.001, ** p$<$0.01 and * p$<$0.05. 
\item This table reports regression results of expected spending growth on perceived income risks (incvar for nominal, rincvar for real).}\end{flushleft}
\end{threeparttable}
\end{adjustbox}
\end{table}


\clearpage


\begin{table}[p]
\centering
\begin{threeparttable}
\caption{Estimated subjective risk perceptions}
\label{tab:PRMarkovEst}
\begin{tabular}{lr}
\toprule
{} &  baseline \\
\midrule

$std(\tilde\sigma)$&     1.203 \\
$q$                     &     0.565 \\
$p$                     &     0.565 \\
$\tilde \sigma^l_\psi$   &     0.897 \\
$\tilde \sigma^l_\theta$ &     0.021 \\
$\tilde \sigma^h_\psi$   &     1.140 \\
$\tilde \sigma^h_\theta$  &     0.027 \\
\bottomrule
\end{tabular}
	\begin{flushleft}
		{\footnotesize This table reports estimates of the parameters for the 2-state Markov switching model of subjective risk perceptions. Risks are at the annual frequency. }
			\end{flushleft}
\end{threeparttable}
\end{table}


\clearpage\begin{table}[p]
\begin{threeparttable}
\caption{Model parameters}
\label{tab:calibration}
\begin{tabular}{llll}
\hline 

block             & parameter name              & values & source                               \\
\hline 

risk              & $\sigma_\psi$               & 0.10   & Median estimates from the literature \\
risk              & $\sigma_\theta$             & 0.15    & Median estimates from the literature \\
risk              & $U2U$                       & 0.18   & Median estimates from the literature \\
risk              & $E2E$                       & 0.96   & Median estimates from the literature \\
\hline 

initial condition & $\sigma_\psi^{\text{init}}$ & 0.629  & Estimated for age 25 in the 2016 SCF \\
initial condition & bequest ratio               & 0      & assumption                           \\
\hline 

life cycle        & $T$                         & 40     & standard assumption                  \\
life cycle        & $L$                         & 60     & standard assumption                  \\
life cycle        & $1-D$                       & 0.994  & standard assumption                  \\
\hline 

preference        & $\rho$                      & 1      & standard assumption                  \\
preference        & $\beta$                     & 0.98   & standard assumption                  \\
\hline 

policy            & $\mathbb{S}$                & 0.65   & U.S. average                         \\
policy            & $\lambda$                   & N/A      & endogenously determined              \\
policy            & $\lambda_{SS}$              & N/A      & endogenously determined              \\
policy            & $\mu$                       & 0.15   & U.S. average                         \\
\hline 

production        & $W$                         & 1      & target values in steady state        \\
production        & K2Y ratio                   & 3      & target values in steady state        \\
production        & $\alpha$                    & 0.33   & standard assumption                  \\
production        & $\delta$                    & 0.025  & standard assumption          \\
\hline 
\end{tabular} 
	\begin{flushleft}
		{\footnotesize This table reports parameters used in the benchmark objective model. All parameters, whenever relevant, are at the annual frequency.}
			\end{flushleft}
\end{threeparttable}
\end{table}