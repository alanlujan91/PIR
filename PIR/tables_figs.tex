                        \newpage 
   
  \section*{Tables and Figures} 
    
    %%%%%%%%%%%%%%%%%%%%%%%%%%%%%
    % figures 
    %%%%%%%%%%%%%%%%%%%%%%%%%%%%%
    
        \begin{figure}[!ht]
    \caption{Dispersion in Unexplained Perceived Income Risks}
    \label{fig:histmoms}
	\begin{center}
    		\adjustimage{max size={0.4\linewidth}{0.3\paperheight}}{figures/joy_incstd.jpg}
    		\adjustimage{max size={0.4\linewidth}{0.3\paperheight}}{figures/joy_rincstd.jpg}
\end{center}
    \begin{flushleft}Note: this figure plots the distributions of residuals of the perceived standard deviation of 1-year-ahead earning growth in nominal (left) and real terms (right) after controlling age, age polynomial, gender, education, type of work, and time fixed effects. The real risk is the sum of the perceived risk of nominal income and inflation uncertainty.\end{flushleft}
    	% average PR: 2.1% in std; 10/90 IQR: 3.2% in std
    \end{figure}
    
    \clearpage
    
    \begin{figure}[!ht]
    	\caption{Realized and Perceived Income Risks over the Life Cycle}
    	\label{fig:age_compare}
    	\begin{center}\adjustimage{max size={0.7\linewidth}}{figures/real_log_wage_shk_gr_by_age_edu_gender_compare.png}\end{center}
    	\begin{flushleft}Note: this figure plots average realized income volatility and perceived risks of different age groups. The realized income volatility is approximated by the cross-sectional standard deviation of log changes in unexplained income residuals within age/education/gender group based on PSID.\end{flushleft}
    \end{figure}

    \clearpage
    \begin{figure}[!ht]
    	\caption{Perceived Earning Risks by Earning Decile}
    	\label{fig:barplot_byinc}
    	\begin{center}\adjustimage{max size={0.7\linewidth}}{figures/boxplot_rvar_earning.png}\end{center}
    \begin{flushleft}Note: this figure plots average perceived income risks by the decile of annual earning of the same individual.\end{flushleft}
    \end{figure}
    
    
    \clearpage
    \begin{figure}[!ht]
    	\caption{Perceived and Realized Earning Risks}
    	\label{fig:ts_compare}
    	\begin{center}
    	\adjustimage{max size={0.5\linewidth}}{figures/real_volatility_compare.png}
    		\vbiskip
    		\adjustimage{max size={0.5\linewidth}}{figures/real_permanent_compare.png}
    	\vbiskip
    	\adjustimage{max size={0.5\linewidth}}{figures/real_transitory_compare.png}
    	\end{center}
    \begin{flushleft}Note: this figure plots median perceived income risks in the whole SCE sample against the realized volatility, permanent, and transitory risks over the same period. The realized risks are first estimated monthly from SIPP and then aggregated into annual frequency. Specifically, the permanent risks are the sum of monthly permanent risks and the annual transitory risks are the simple average over the corresponding 12 months.\end{flushleft}
    \end{figure}
    
  
    \clearpage
    \begin{figure}[!ht]
      \caption{Recent Labor Market Conditions and Perceived Risks}
    \label{fig:ts_he}
    	\begin{center}\adjustimage{max size={\linewidth}}{figures/tsMean3mvrvar_he.jpg}
    \end{center}
    \begin{flushleft}Note: recent labor market outcome is measured by hourly wage growth (YoY). The 3-month moving average is plotted for both series.\end{flushleft}
    \end{figure}
    
     
    \clearpage
    \begin{figure}[!ht]
      \caption{Expected and Realized Job-separation rate}
    \label{fig:srate_compare}
    	\begin{center}\adjustimage{max size=0.7\linewidth}{figures/seperation_rate_1y.png}\end{center}
    \begin{flushleft}Note: realized job separation rate is computed from CPS. Both are expressed as Poisson arrival rates in continuous time with one month as the unit of time. The 5-month moving average expected rate is plotted.\end{flushleft}
    \end{figure}
    
    
 \newpage
 
\begin{figure}[!ht]
	\caption{Experience and Perceived Income Risk}
	\label{fig:var_experience_data}
	\begin{center}
		\adjustimage{max size={0.7\linewidth}{0.4\paperheight}}{figures/experience_var_var_data.png}
	\adjustimage{max size={0.7\linewidth}{0.4\paperheight}}{figures/experience_ue_var_data.png}
\end{center}
\begin{flushleft}Note: the experienced income volatility is the cross-sectional variance of log change in income residuals estimated using a sub sample restricted to the lifetime of a particular group. For instance, the life experience of a 25-year old till 2015 spans from 1990-2015. The perceived income risk is the average across all individuals from the cohort in that year. Cohorts are time/year-of-birth specific and all cohort sized 30 or smaller are excluded.\end{flushleft}
\end{figure}



 \newpage
 
 \begin{comment}
 
\begin{figure}[!ht]
	\caption{Consumption functions under objective and subjective risk profiles}
	\label{fig:comparison1}
	\begin{center}
		\adjustimage{max size={0.8\linewidth}{0.5\paperheight}}{figures/comparison1.png}
\end{center}
\begin{flushleft}Note: this figure plots an example of age-specific optimal consumption policies under objective and subjective state-dependent income risk profiles, respectively. The two models use the exact same parameter configurations except for the subjective model having agents to stochatically draw a low or high risk perception. Both the average size of transitory and permanent risks are kept equal to that in the objective model.\end{flushleft}
\end{figure}

\newpage 

\begin{figure}[!ht]
	\caption{Consumption functions under objective and state-dependent risk profiles}
	\label{fig:comparison2}
	\begin{center}
		\adjustimage{max size={0.8\linewidth}{0.5\paperheight}}{figures/comparison2.png}
\end{center}
\begin{flushleft}Note: this figure plots age-specific optimal consumption policies under objective and subjective extrapolation model. The former model includes persistent unemployment risks while the latter maintains the same transition probability of employment status but allows the risk perceptions to be dependent upon the employment status of the individual, i.e. unemployed perceive income risks to be higher than the employed.\end{flushleft}
\end{figure}

 \end{comment}
%%%%%%%%%%%%%%%%%%%%%%%%%%%%
%% comment out the below figures 
%%%%%%%%%%%%%%%%%%%%%%%%%%%%%%%%

\begin{comment}

\clearpage
\begin{figure}[!ht]
	\caption{Experience and Perceived Income Risk: Permanent and Transitory}
	\label{fig:experience_var_per_tran_var_data}
	\begin{center}
		\adjustimage{max size={0.6\linewidth}{0.3\paperheight}}{../../Graphs/ind/experience_var_permanent_var_data.png}
		\adjustimage{max size={0.6\linewidth}{0.3\paperheight}}{../../Graphs/ind/experience_var_transitory_var_data.png}
		\adjustimage{max size={0.6\linewidth}{0.3\paperheight}}{../../Graphs/ind/experience_var_ratio_var_data.png}
\end{center}
\begin{flushleft}Note: experienced permanent (transitory) volatility is average of the estimated risks of the permanent (transitory) component of a particular year-cohort sample. The perceived income risk is the average across all individuals from the cohort in that year.\end{flushleft}
\end{figure}

\clearpage


\begin{figure}[!ht]
	\caption{Experience and Perceived Income Risk: Aggregate and Idiosyncratic}
	\label{fig:experience_id_ag_data}
	\begin{center}\adjustimage{max size={0.4\linewidth}{0.3\paperheight}}{../../Graphs/ind/experience_id_gr_var_data.png} 
		\adjustimage{max size={0.4\linewidth}{0.3\paperheight}}{../../Graphs/ind/experience_var_id_var_data.png}
	\adjustimage{max size={0.4\linewidth}{0.3\paperheight}}{../../Graphs/ind/experience_ag_gr_var_data.png}
	\adjustimage{max size={0.4\linewidth}{0.3\paperheight}}{../../Graphs/ind/experience_var_ag_var_data.png}
\adjustimage{max size={0.4\linewidth}{0.3\paperheight}}{../../Graphs/ind/experience_ue_var_data.png}
\adjustimage{max size={0.4\linewidth}{0.3\paperheight}}{../../Graphs/ind/experience_ue_var_var_data.png}
\end{center}
\begin{flushleft}Note: experienced idiosyncratic income shocks are approximated as the cohort-specific average unexplained income residuals from a regression controlling time-fixed and education/time fixed effect.  Aggregate shock is approixmated as the average income change explained by the two effects.  The perceived income risk is the average across all individuals from the cohort in that year.\end{flushleft}
\end{figure}

\clearpage

\begin{figure}[!ht]
	\caption{Attribution and Parameter Uncertainty}
	\label{fig:corr_var}
	\begin{center}\adjustimage{max size={0.9\linewidth}{0.4\paperheight}}{../../Graphs/theory/corr_var.jpg}\end{center}
\begin{flushleft}Note: this figure illustrates how parameter uncertainty changes with the subjective correlation of one's own income and others'.\end{flushleft}
\end{figure}

\clearpage
\begin{figure}[!ht]
	\caption{Experienced Volatility and Perceived Risk}
	\label{fig:var_experience_var}
	\begin{center}\adjustimage{max size={0.9\linewidth}{0.4\paperheight}}{../../Graphs/theory/var_experience_var.jpg}\end{center}
\begin{flushleft}Note: this figure illustrates the relationship between experienced volatility and perceived income income risk under different attributions.\end{flushleft}
\end{figure}

\clearpage
\begin{figure}[!ht]
	\caption{Attribution Function}
	\label{fig:theta_corr}
	\begin{center}\adjustimage{max size={0.9\linewidth}{0.4\paperheight}}{../../Graphs/theory/theta_corr.jpg}\end{center}
	\floatfoot{Note: this figure illustrates the parameterized attribution function under different degree of attribution error governed by $\theta$.}
\end{figure}

\clearpage
\begin{figure}[!ht]
	\caption{Current Income and Perceived Risk}
	\label{fig:var_recent}
	\begin{center}\adjustimage{max size={0.9\linewidth}{0.4\paperheight}}{../../Graphs/theory/var_recent.jpg}\end{center}
\begin{flushleft}Note: this figure plots the theoretical prediction of the relationship between current income and perceived income risks.\end{flushleft}
\end{figure}


\clearpage
\begin{figure}[!ht]
	\caption{Simulated Income Profile of Perceived Risk}
	\label{fig:var_recent_sim}
	\begin{center}\adjustimage{max size={0.9\linewidth}{0.4\paperheight}}{../../Graphs/theory/var_recent_sim.jpg}\end{center}
\begin{flushleft}Note: this figure plots the simulated relationship between current income and perceived income risks under the theory.\end{flushleft}
\end{figure}

    
    \clearpage
    \begin{figure}[!ht]
    	\caption{Simulated Age Profile of Perceived Risk}
    	\label{fig:var_age_sim}
    	\begin{center}\adjustimage{max size={0.9\linewidth}{0.4\paperheight}}{../../Graphs/theory/var_age_sim.jpg}\end{center}
    	\floatfoot{Note: this figure plots the simulated relationship between age and perceived income risks.}
    \end{figure}
    
    \clearpage
    \begin{figure}[!ht]
    	\caption{Simulatd Average Labor Market and Perceived Risk}
    	\label{fig:recent_change_var_sim}
    	\begin{center}   		
    		\adjustimage{max size={0.9\linewidth}{0.4\paperheight}}{../../Graphs/theory/var_recent_change_sim.jpg} \\
    	\adjustimage{max size={0.9\linewidth}{0.4\paperheight}}{../../Graphs/theory/var_recent_change_sim2.jpg}
\end{center}
\begin{flushleft}Note: this figure plots the simulated relationship between average perceived risks and average income changes with/without attribution errors (the upper panel) and under aggregate/idiosyncratic risks (the bottom panel). 
\end{flushleft}
    
    \end{figure}
    
    
   
\end{comment}



%%%%%%%%%%%%%%%%%%%%%%%%%%%%%%%%%%%%%%% % tables   
%%%%%%%%%%%%%%%%%%%%%%%%%%%%%%%%%%%%%
\clearpage

\begin{table}[ht]
\centering
\begin{adjustbox}{width={\textwidth}}
\begin{threeparttable}
\caption{Current Labor Market Conditions and Perceived Income Risks}
\label{macro_corr_he}
\begin{tabular}{ccccccl}
\toprule
{} &  mean:var &  mean:iqr & mean:rvar & median:var & median:iqr & median:rvar \\
\midrule
0 &   -0.28** &  -0.42*** &  -0.48*** &      -0.16 &      -0.16 &    -0.53*** \\
1 &  -0.44*** &  -0.54*** &  -0.51*** &      -0.02 &      -0.02 &    -0.53*** \\
2 &  -0.39*** &  -0.44*** &  -0.43*** &      -0.05 &        0.0 &    -0.45*** \\
3 &  -0.44*** &  -0.47*** &  -0.41*** &      -0.09 &      -0.06 &     -0.5*** \\
4 &   -0.29** &  -0.38*** &  -0.32*** &      -0.19 &      -0.14 &     -0.5*** \\
\bottomrule
\end{tabular}
\begin{flushleft}
\item *** p$<$0.001, ** p$<$0.01 and * p$<$0.05.
\item This table reports correlation coefficients between different perceived income moments(inc for nominal
and rinc for real) at time
$t$ and the quarterly growth rate in hourly earning at $t,t-1,...,t-k$.
\end{flushleft}
\end{threeparttable}
\end{adjustbox}
\end{table}
\clearpage

	
	\begin{table}[ht]
		\centering
		\begin{adjustbox}{width=0.9\textwidth}
			\begin{threeparttable}
			\caption{Average Perceived Risks and State Labor Market}
			\label{macro_corr_he_state}
			\begin{tabular}{lllll}
					\hline 
				& (1)                & (2)                & (3)               & (4)               \\
				& log perceived risk & log perceived risk & log perceived iqr & log perceived iqr \\
				\hline 
				Wage Growth (Median) & -0.05***           &                    & -0.03***          &                   \\
				& (0.01)             &                    & (0.01)            &                   \\
				&                    &                    &                   &                   \\
				UE (Median)          &                    & 0.04*              &                   & 0.04***           \\
				&                    & (0.02)             &                   & (0.01)            \\
				&                    &                    &                   &                   \\
					\hline 
				Observations         & 3589               & 3589               & 3596              & 3596              \\
				R-squared            & 0.021              & 0.019              & 0.025             & 0.027             \\
				\hline      
			\end{tabular}

				\begin{flushleft}
					\item *** p$<$0.001, ** p$<$0.01 and * p$<$0.05.
					\item This table reports regression coefficient of the average perceived income risk of each state in different times on current labor market indicators, i.e. wage growth and unemployment rate. Montly state wage series is from Local Area Unemployment Statistics (LAUS) of BLS. Quarterly state unemployment rate is from Quarterly Census of Employment and Wage (QCEW) of BLS. 
				\end{flushleft}
			\end{threeparttable}
		\end{adjustbox}
	\end{table}


\clearpage

\begin{table}
	\centering
	\caption{Extrapolation from Recent Experience}
	\label{extrapolation}
	\adjustbox{max height=0.5\textheight, max width=\textwidth}{ 
	\begin{tabular}{lllllllllll}
		\hline 
		& (1)       & (2)       & (3)       & (4)       & (5)        & (6)        & (7)        & (8)        & (9)        & (10)       \\
		\hline 
		income shock squared                  & 0.0225*** & 0.0222*** & 0.0217*** & 0.0207*** & 0.000773   & 0.00205*** & 0.000566   & 0.00183*** & 0.000614   & 0.00184*** \\
		& (0.00562) & (0.00570) & (0.00562) & (0.00564) & (0.000743) & (0.000516) & (0.000744) & (0.000515) & (0.000745) & (0.000516) \\
		&           &           &           &           &            &            &            &            &            &            \\
		recently unemployed                   &           &           &           & 0.511*    & 0.228***   & 0.0895***  &            &            &            &            \\
		&           &           &           & (0.260)   & (0.0330)   & (0.0200)   &            &            &            &            \\
		&           &           &           &           &            &            &            &            &            &            \\
		unemployed since m-8&           &           &           &           &            &            & 0.161***   & 0.0783***  &            &            \\
		&           &           &           &           &            &            & (0.0207)   & (0.0121)   &            &            \\
		&           &           &           &           &            &            &            &            &            &            \\
		unemployed since y-1&           &           &           &           &            &            &            &            & 0.138***   & 0.0701***  \\
		&           &           &           &           &            &            &            &            & (0.0193)   & (0.0113)   \\
		Observations                          & 3662      & 3662      & 3662      & 3662      & 3701       & 1871       & 3701       & 1871       & 3701       & 1871       \\
		R-squared                             & 0.004     & 0.013     & 0.016     & 0.017     & 0.015      & 0.030      & 0.019      & 0.041      & 0.016      & 0.039      \\
		\hline 
	\end{tabular}
}
	\begin{flushleft}\item Standard errors are clustered by household. *** p$<$0.001, ** p$<$0.01 and * p$<$0.05. 
\item This table reports regression of perceived risks and perceived unemployment risks on recent experiences of income volatility and the dummy indicating if the individual has recently experienced an unemployment. 
\end{flushleft}
\end{table}


\clearpage

\begin{table}[p]
\centering
\begin{adjustbox}{width=\textwidth}
\begin{threeparttable}
\caption{Perceived Income Risks, Experienced Volatility and Individual Characteristics}
\label{micro_reg}\begin{tabular}{lllllll}
\toprule
{} & incvar I & incvar II & incvar III & incvar IIII & incvar IIIII & incvar IIIIII \\
                    &          &           &            &             &              &               \\
\midrule
IdExpVol            &  4.58*** &   2.23*** &    2.69*** &     2.75*** &      2.95*** &       2.94*** \\
                    &   (0.33) &    (0.36) &     (0.39) &      (0.39) &       (0.38) &        (0.39) \\
AgExpVol            &     0.04 &   0.28*** &    0.34*** &     0.32*** &      0.18*** &       0.20*** \\
                    &   (0.04) &    (0.04) &     (0.05) &      (0.05) &       (0.05) &        (0.05) \\
AgExpUE             &  0.14*** &   0.08*** &     0.05** &       0.05* &        0.04* &        0.05** \\
                    &   (0.02) &    (0.02) &     (0.02) &      (0.02) &       (0.02) &        (0.02) \\
age                 &          &  -0.02*** &   -0.02*** &    -0.02*** &     -0.02*** &      -0.02*** \\
                    &          &    (0.00) &     (0.00) &      (0.00) &       (0.00) &        (0.00) \\
gender=male         &          &           &   -0.36*** &    -0.35*** &     -0.32*** &      -0.30*** \\
                    &          &           &     (0.02) &      (0.02) &       (0.02) &        (0.02) \\
nlit\_gr=low nlit    &          &           &    0.09*** &     0.09*** &      0.10*** &       0.09*** \\
                    &          &           &     (0.02) &      (0.02) &       (0.02) &        (0.02) \\
parttime=yes        &          &           &            &             &        -0.01 &         -0.02 \\
                    &          &           &            &             &       (0.02) &        (0.02) \\
selfemp=yes         &          &           &            &             &      1.25*** &      -0.00*** \\
                    &          &           &            &             &       (0.03) &        (0.00) \\
UEprobAgg           &          &           &            &             &              &       0.02*** \\
                    &          &           &            &             &              &        (0.00) \\
UEprobInd           &          &           &            &             &              &       0.02*** \\
                    &          &           &            &             &              &        (0.00) \\
HHinc\_gr=low income &          &           &            &             &      0.16*** &       0.16*** \\
                    &          &           &            &             &       (0.02) &        (0.02) \\
educ\_gr=high school &          &           &            &    -0.10*** &     -0.13*** &      -0.09*** \\
                    &          &           &            &      (0.02) &       (0.02) &        (0.02) \\
educ\_gr=hs dropout  &          &           &            &        0.08 &         0.11 &       0.29*** \\
                    &          &           &            &      (0.11) &       (0.11) &        (0.11) \\
N                   &    41422 &     41422 &      34833 &       34833 &        33480 &         29687 \\
R2                  &     0.01 &      0.02 &       0.04 &        0.04 &         0.11 &          0.06 \\
\bottomrule
\end{tabular}
	\begin{flushleft}
\item Standard errors are clustered by household. *** p$<$0.001, ** p$<$0.01 and * p$<$0.05. 
\item This table reports results associated a regression of logged perceived income risks (incvar) on logged idiosyncratic($\text{IdExpVol}$), aggregate experienced volatility($\text{AgExpVol}$), experienced unemployment rate (AgExpUE), and a list of household specific variables such as age, income, education, gender, job type and other economic expectations.
\end{flushleft}
\end{threeparttable}
\end{adjustbox}
\end{table}
\clearpage

\begin{table}[p]
\centering
\begin{adjustbox}{width={0.9\textwidth}}
\begin{threeparttable}
\caption{Perceived Income Risks and Household Spending Plan}
\label{spending_reg}
\begin{tabular}{lllllll}
	\hline 
	& (1)      & (2)      & (3)      & (4)      & (5)      & (6)      \\
		\hline 
	perceived earning risk           & 8.394*** & 8.399*** & 3.642*** & 3.243*** &          &          \\
	& (1.175)  & (1.176)  & (0.533)  & (0.537)  &          &          \\
	&          &          &          &          &          &          \\
	perceived earning risk (nominal) &          &          &          &          & 3.656*** &          \\
	&          &          &          &          & (0.990)  &          \\
	&          &          &          &          &          &          \\
	perceived ue risk                &          &          &          &          &          & 0.353*** \\
	&          &          &          &          &          & (0.0553) \\
		\hline 
	R-squared                        & 0.0010 & 0.00282  & 0.928    & 0.928    & 0.941    & 0.633    \\
	Sample Size                      & 53178    & 53178    & 53178    & 53178    & 54584    & 6269     \\
	Time FE                          & No       & Yes      & No       & Yes      & Yes      & No       \\
	Individual FE                    & Yes       & No       & Yes      & Yes      & Yes      & Yes     \\
		\hline 
\end{tabular}
	\begin{flushleft}
		{\footnotesize\item Standard errors are clustered by household. *** p$<$0.001, ** p$<$0.01 and * p$<$0.05. 
\item This table reports regression results of expected spending growth on perceived income risks (incvar for nominal, rincvar for real).}\end{flushleft}
\end{threeparttable}
\end{adjustbox}
\end{table}

\clearpage


\begin{table}[p]
\centering
\begin{threeparttable}
\caption{Estimated subjective risk perceptions}
\label{tab:PRMarkovEst}
\begin{tabular}{lr}
\hline 
&  baseline \\
\midrule
$q$                     &     0.792 \\
$p$                     &     0.792 \\
$\tilde\sigma_\psi^l$   &     0.456 \\
$\tilde\sigma_\theta^l$ &     0.089 \\
$\tilde\sigma_\psi^h$   &     0.456 \\
$\tilde\sigma_\theta^l$ &     0.089 \\
$\tilde \mho^l$         &     0.072 \\
$\tilde E^l$            &     0.703 \\
$\tilde \mho^h$        &     0.359 \\
$\tilde E^h$ &     0.514 \\
\hline 
\end{tabular}
	\begin{flushleft}
		{\footnotesize This table reports estimates of the parameters for the 2-state Markov switching model of subjective risk perceptions. Risks are at the monthly frequency.  }
			\end{flushleft}
\end{threeparttable}

\end{table}
