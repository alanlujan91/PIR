\hypertarget{unemployment-risk-perceptions}{%
\subsection{Unemployment risk perceptions}\label{unemployment-risk-perceptions}}

The analysis so far only focuses on wage risks conditional on staying in the same job. But it admittedly only constitutes a part of the income risks, since major labor market transitions such as job loss and switching usually result in more significant changes in labor income and affects a household's welfare\footnote{\cite{low2010wage}, \cite{davis2011recessions}.}. Unemployment risks are usually another central input of the incomplete-market macroeconomic models.\footnote{For examples, see \cite{krueger2016macroeconomics} and \cite{bayer2019precautionary}, etc.} And similar to the approach with wage risks, the common practice in these models is to model the process of labor market transitions based on externally estimated stochastic process\footnote{The exceptions are models endogenizing job search \& match mechanisms, such as \cite{ravn2017job}, \cite{ravn2021macroeconomic}, \cite{mckay2017time}, in which typically job-separation rates remains exogenous and externally calibrated.}. This section shows that the survey-reported expectations of job separation/finding probabilities on average keep track with realized aggregate dynamics computed from the panel data, while masks a sizable degree of heterogeneity, which is assumed in standard models.  


For a fair comparison between perceptions and realizations which are regarding different horizons, I cast both probabilities into a continuous-time rate for a Poisson point process. Specifically, for the expectation, let the reported probability of separating from the current job in the next 12 months be $P_{i,t}(ue_{t+12}|e_t)$, then the corresponding monthly Poisson rate of job-separation $E_{i,t}(s_{t+1})$ is $- log(1-P_{i,t}(u_{t+12}|e_t))/12$\footnote{This follows from the following mathematical fact: for a continuous-time Poisson process with an event rate of $\theta$, the arrival probability over a period of $\Delta t$ units of time is equal to $1-exp^{-\theta \Delta t}$.}. With the realized month-to-month flow rate estimated from CPS $P(ue_{t+1}|e_t)$ , the corresponding realized Poisson rate  $s_{t+1}$ is $-log(1-P(ue_{t+1}|e_t))$. 


Figure \ref{fig:srate_compare} plots the converted job-separation/finding expectations and their respective realizations against each other. A few important patterns emerge. In addition, I plot the 25 and 75 percentile of the expectations across all survey respondents around its population average. A number of straightforward findings emerge. First, although the two series are independently constructed of each other, on average, perceptions did track the aggregate realizations relatively well. The most notable deviation between the belief and realization was during March 2020, which marked the unprecedented increase in the one-month job separation\footnote{The March observation was dropped in the graph, otherwise, it overshadows all other observations in the sample.} and a dramatic decrease in job finding. Second, however, as shown by the wide 25/75 inter-range-percentile around mean expectations, individual respondents vastly disagree on their individual separation and finding probabilities. Since the question in the survey regards the individual-specific transitions, it is most reasonable to assume that this reflects the unobserved heterogeneity or information available to their individual status, which economists cannot directly observe. 



\begin{center}
[FIGURE \ref{fig:srate_compare} HERE]
\end{center}



