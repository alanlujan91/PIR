
    \hypertarget{introduction}{%
\section{Introduction}\label{introduction}}

Income risks matter for both individual behaviors and aggregate
outcomes. With identical expected income and homogeneous risk
preferences, different degrees of risks lead to different
saving/consumption and portfolio choices. This is well understood in
models in which either the prudence in the utility function
(\cite{kimball1990precautionary}, \cite{carroll2001liquidity}) or occasionally binding constraint induces precautionary
savings or self-insurance. It is widely accepted based on various empirical research that
idiosyncratic income risks are at most partially insured
(\cite{blundell_consumption_2008}), such market incompleteness leads
to ex-post wealth inequality\footnote{\cite{ aiyagari1994uninsured,huggett1996wealth,carroll1997nature,krusell1998income}.} and different degrees of marginal
propensity to consume (MPC) (\cite{krueger2016macroeconomics, carroll2017distribution}). This also changes the mechanisms via which macroeconomic policies take into effect\footnote{\cite{krueger2016macroeconomics}, \cite{kaplan2018monetary}, \cite{auclert2019monetary}.}. Furthermore, the aggregate movements in the degree of idiosyncratic labor risks drive time-varying precautionary saving motives, as another source of business cycle fluctuations.\footnote{ \cite{challe2016precautionary, mckay2017time,heathcote2018wealth, kaplan2018microeconomic,den2018unemployment,bayer2019precautionary, acharya2020understanding,ravn2021macroeconomic,harmenberg2021consumption}.}

The size and the nature of the income risks are one of the central inputs in this class of incomplete-market macroeconomic models. One common practice prevailing in this literature thus far is that economists typically approximate/estimate risks under a specified income process, relying upon the cross-sectional dispersion in income realizations, and then treat the estimates
as the true model parameters known by the agents making decisions in the
model.\footnote{Some recent examples include \cite{krueger2016macroeconomics}, \cite{bayer2019precautionary}, \cite{kaplan2018monetary}.}
  
  
But this estimation practice has limitations. Economists who attempt to approximate the real size and nature of unexpected income shocks and risks as perceived by the agents may very likely face omitted variables/unobserved heterogeneity and model mis-specification regarding its heterogeneity in risks. The intuition behind this is simple: some information, either intrinsic heterogeneity of each individual or advance information that enters an agent's information set from time to time and is used to forecast her income, is not directly observable by economists. Therefore, what is a shock as approximated by economists may be expected by agents already, and what is considered as the risk is not one from the agents' point of view, either.



%To the extent that agents make decisions based on their \emph{respective} perceptions, understanding the \emph{perceived} income risk profile are the keys to explaining both micro and macro economic dynamics.

This paper attempts to address these issues by utilizing the recently available density forecasts of labor income surveyed by New York Fed's Survey of Consumer Expectation (SCE). The most important novelty of this paper compared to previous work studying partial insurance with expectational surveys \footnote{For instance, \cite{pistaferri_superior_2001}, \cite{kaufmann_disentangling_2009}.} is that I particularly use the density survey which contains directly perceived risks.  What is special about the density survey is that agents are asked to provide histogram-type forecasts of their wage growth over the next 12 months, together with a set of expectational questions about the macroeconomy. When the individual density forecast is available, a parametric density estimation can be made to obtain the individual-specific subjective distribution. And the second moment, namely the implied variance of the subjective distribution, allow me to directly characterize the perceived risk profile without relying on external estimates from cross-sectional microdata. This provides the first-hand measure of perceived income risks that are truly relevant to individual decisions.

With the individual-specific reported perceived risk (PR) available, I can directly examine the cross-sectional heterogeneity in PR even within groups that conventional estimating methods assume
to share the same degree of risks, such as education, age, and gender. I confirm that heterogeneity in PR does reflect between-group differences in idiosyncratic income risks, as revealed by estimates using panel data. For instance, younger, low-income, females, with low education with more volatile income growth also perceive higher risks. But despite controlling for these observable factors, people with the same observable characteristics are still widely dispersed in risk perceptions. A dominant
share of the heterogeneity can be only attributed to unobserved heterogeneity/information. This suggests the importance of incorporating heterogeneity in income risks beyond a limited number of dimensions, such as education and age, as the standard practice in the literature.


These evidence motivate me to utilize survey-implied risks as truly perceived by agents to calibrate income risks in a standard incomplete market, overlapping-generation, and general equilibrium model to quantify these effects. The objective/benchmark model blends \cite{huggett1996wealth}, the income structure of \cite{carroll1997nature}, and persistent unemployment spells and unemployment benefits, a la \cite{krueger2016macroeconomics} and \cite{carroll2017distribution}. In comparison with conventional practice, I show how calibrating risks using surveyed perceptions helps explain a number of well-documented discrepancies between standard model prediction and that seen in the data: the concentration of households with little liquid wealth, a large fraction of agents with high $MPC$, and more wealth inequality. 

The intuitions behind these results are straightforward. Lower perceived risks imply lower precautionary savings. In addition, allowing for heterogeneity in risk perceptions induces a straightforward increase in wealth inequality, simply because different risks induce different optimal savings. 


%The intuitions behind these results are straightforward. Lower perceived risks than standard estimates used to parameterize the model implies lower precautionary savings, while the state-dependence and extrapolation both induce additional precautionary savings, as shown by \cite{caballero1990consumption}. Therefore, the quantitative implications of the survey-implied risk perceptions on the level of savings depend on the counterbalance of the two forces. The effect on wealth inequality is less ambiguous. Allowing for heterogeneity in risk perceptions induce straightforward increase in wealth inequality, simply because different risks induce different optimal savings. 



%On this front is to establish a unified framework for perceived income risks under different possible income processes seen in the macro literature. Under a clearly specified income process, I can examine to what extent the perceived income risks align with a number of benchmark predictions under full-information rational expectation(FIRE) and with a list of empirically documented facts regarding the income risk dynamics. For instance, is there a large degree of dispersion in risk perceptions among agents who the modelers assume face the same level of risks? Other questions I use the survey to answer include: are perceived risks state-dependent and counter-cyclical? Do people extrapolate and overreact to recent experiences when forming risk perceptions? Does the perceived risk reflect a reasonably good understanding of the income risks of different nature? The answers to all of these questions are yes. 

%Individuals of varying characteristics face potentially different income processes. Even under the same income process, the realizations of income differ across agents due to differences in realized shocks. In addition to the fact that realized income is not observed in these surveys, this makes it additionally challenging to undertake comparisons between perceptions and the underlying process in a similar manner as for expectations about macroeconomic variables such as inflation. A clear comparison of such spirit is also possibly sensitive to the consistency between the frequency of the reported income perception and the frequency of the underlying income process, i.e.~the time aggregation problem. Besides, I also explicitly take into account the presence of the superior information problem extensively discussed in the literature.

On the flip side, there is mounting evidence in macroeconomics that people form expectations in ways deviating from full-information rational expectation (FIRE) \footnote{For instance, \cite{mankiw2003disagreement}, \cite{reis2006inattentive}, \cite{coibion2012can}, \cite{wang2021infvar}, although most of these evidence are based on macroeconomic expectations such as inflation.} leading to perennial expectational heterogeneity across agents. Therefore, it is worth considering the robustness of using survey-implied risk perceptions to generate model implications. As this paper cannot fully separately identify the behavioral bias in risk perceptions, I proceed with an additional experiment model, allowing the risk perceptions (subjective risks) to be different from the underlying income process (objective risks). This experiment model kills two birds with one stone. On one hand, it serves as a robustness check with an alternative model assumption deviating from FIRE. On the other hand, it can be invoked as an intermediary to break down the model implications into two channels: one via ex-anted saving behaviors resulting from risk perceptions and the other via ex-post realized income inequality. 

The baseline model shows that survey-reported income risks can be directly used as reasonable inputs in structural macro models calibration featuring belief heterogeneity. As an extension, I also undertake additional estimation procedures to uncover the stochastic changes of each individual's PR due to reasons such as new information, which is not directly observed by economists with the surveys. The central idea of this estimation is to treat the survey-reported perceived risks as noisy signals of a number of hidden stochastic states of belief subject to measurement errors. In the two-state regime scenario, this essentially is to estimate a modified 2-Markov regime-switching model in the spirit of \cite{hamilton1989new}. As a methodological contribution, my exercises provide a potentially generalizable approach to utilizing expectational surveys in macroeconomic models which may involve unobserved information and measurement errors. 

\subsection{Derivative results}

The main body of the paper primarily focuses on cross-sectional heterogeneity in idiosyncratic risks and its downstream macroeconomic implications. But, in the Appendix, I report additional results from inspecting other drivers of PR. For instance, Section \ref{counter-cyclicality-of-perceived-risk} reports ample evidence on the counter-cyclical patterns of PR over business cycles. PR negatively correlate with the nationwide/regional labor market conditions.\footnote{This bears similarities to but the important difference with a few previous studies that document the counter-cyclicality of income risks estimated by cross-sectional microdata (\cite{guvenen2014nature},\cite{catherine_countercyclical_2019})} Appendix \ref{experiences-and-perceived-risk} shows that both recent labor market outcomes and the historical experience of income volatility affect individual PRs. I provide evidence for both extrapolation and experience-based learning in risk perception formation.\footnote{I further explored these mechanisms in risk perception formation in a companion paper} Recent unemployed and higher recent wage volatility is associated with higher perceived risks. Reminiscent of the findings by \cite{malmendier2015learning} and \cite{kuchler2019personal} in other contexts, higher experienced volatility and experience of negative labor market conditions are found to be important explaining cross-generational differences in PR. 

Appendix \ref{appendix:subjective_model} presents in detail the extension of the standard model with different subjective and objective risks. It features a single deviation by introducing an idiosyncratic subjective state that swings between low and high-risk perceptions, the process of which is estimated from the survey data. This assumption easily accommodates heterogeneity, state-dependence, and extrapolation of risk perceptions. In comparison with the objective model, the subjective model adds a state variable to individuals' consumption problems, and its dynamics also drive the distributional evolution of the economy in wealth. I characterize the economy with a stationary equilibrium. I also explore an extension of the subjective model by assuming the risk perception state depends on employment status of the individuals. 
 
 


%These patterns suggest that individuals have a roughly good yet imperfect understanding of their income risks. Good, in the sense that subjective perceptions are broadly consistent with the realization of cross-sectional income patterns. This is attained in my model because agents learn from past experiences, roughly as econometricians do. In contrast, subjective perceptions are imperfect in that bounded rationality prevents people from knowing about the true size and nature of income shocks as well some parameters of the process perfectly. If hardworking economists equipped with advanced econometric techniques and a large sample of income data do not necessarily specify the income process correctly, it is feasible to admit the agents in the model to be subject to the same difficulty.

%As illustrated by much empirical work of testing the rationality of expectations, it is admittedly challenging to separately account for the differences in perceptions driven by the ``truth'' and the part driven by the pure subjective heterogeneity. The most straightforward way seems to be to treat econometricians' external estimates of the income process as the proxy to the truth, for which the subjective surveys are compared. But this approach implicitly assumes that econometricians correctly specify the model of the income process and ignores the possible ``superior information'' problem that is available only to the people in the sample but not to econometricians. Both my empirical comparisons and model assumptions reconcile these possible challenges by separately characterizing the objective and subjective profile, without taking a strong stance on who is the correct and who is wrong. 

%Finally, the subjective learning model will be incorporated into an otherwise standard life-cycle consumption/saving model with uninsured idiosyncratic and aggregate risks. Experience-based learning makes income expectations and risks state-dependent when agents make dynamically optimal decisions at each point of the time. In particular, higher perceived risks will induce more precautionary saving behaviors. If this perceived risk is state-dependent on recent income changes, it will potentially shift the distribution of MPCs along income decile, therefore, amplify the channels aggregate demand responses to shocks.

\import{./}{1.1literature.tex}

