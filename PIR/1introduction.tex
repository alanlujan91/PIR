
    \hypertarget{introduction}{%
\section{Introduction}\label{introduction}}

Income risks matter for both individual behaviors and aggregate
outcomes. With identical expected income and homogeneous risk
preferences, different degrees of risks lead to different
saving/consumption and portfolio choices. This is well understood in
models in which either the prudence in the utility function
(\cite{kimball1990precautionary}, \cite{carroll2001liquidity}) or occasionally binding constraint induces precautionary
savings or self-insurance. It is widely accepted based on various empirical research that
idiosyncratic income risks are at most partially insured
(\cite{blundell_consumption_2008}), such market incompleteness leads
to ex-post wealth inequality\footnote{\cite{ aiyagari1994uninsured,huggett1996wealth,carroll1997nature,krusell1998income}.} and different degrees of marginal
propensity to consume (MPC) (\cite{krueger2016macroeconomics, carroll2017distribution}). This also changes the mechanisms via which macroeconomic policies take into effect\footnote{\cite{krueger2016macroeconomics}, \cite{kaplan2018monetary}, \cite{auclert2019monetary}.}. Furthermore, the aggregate movements in the degree of idiosyncratic labor risks drive time-varying precautionary saving motives, as another source of business cycle fluctuations.\footnote{ \cite{challe2016precautionary, mckay2017time, kaplan2018microeconomic,den2018unemployment,bayer2019precautionary, acharya2020understanding,ravn2021macroeconomic}.}

The size and the nature of the income risks are one of the central inputs in this class of incomplete-market macroeconomic models. One important assumption prevailing in these models thus far is that agents have a perfect understanding of the
income risks, as the economists assume. Economists typically approximate income risks under a specified income process based on microdata and then treat the estimates
as the true model parameters known by the agents making decisions in the
model. \footnote{Some recent examples include \cite{krueger2016macroeconomics}, \cite{bayer2019precautionary}, \cite{kaplan2018monetary}.}
 But given the mounting evidence that people form expectations in ways
deviating from full-information rational expectation (FIRE) \footnote{For instance, \cite{mankiw2003disagreement}, \cite{reis2006inattentive}, \cite{coibion2012can}, \cite{wang2021infvar}.} leading to perennial
expectational heterogeneity across agents, this
assumption seems to be too stringent. On the flip side of the same coin, economists who attempt to approximate the real size and nature of income shocks as perceived by the agents may also run into problems such as omitted variable and model mis-specification. Both possibilities may result in differences between perceived risks and the standard model assumptions. 

%To the extent that agents make decisions based on their \emph{respective} perceptions, understanding the \emph{perceived} income risk profile are the keys to explaining both micro and macro economic dynamics.

This paper attempts to address these issues by utilizing the recently available density forecasts of labor income surveyed by New York Fed's Survey of Consumer Expectation (SCE). I first characterize the differences between perceived risks reported in the survey and the standard income risk estimates used to parameterize macroeconomic models. My key finding is that perceived earning risks are lower than standard estimates and model parameters, possibly due to either ``superior information'' of the agents compared to economists, or under-perceiving of risks of the agents due to behavioral reasons. This is robust to various income processes used in the literature. Second, I explore a variety of drivers of heterogeneity in risk perceptions, including demographics, recent labor market conditions, and individual and macro experience, as detailed below. Third, I extend a standard incomplete market macro model a la \cite{huggett1993risk} with a subjective risk profile reflecting my empirical findings, allowing the risk perceptions to be different from the underlying income process. I show how such an extension admitting subjective heterogeneity in risk perceptions helps explain a number of the well-documented macroeconomic phenomenon: the concentration of households with little liquid wealth, a large fraction of agents with high $MPC$, and more wealth inequality. 

%On this front is to establish a unified framework for perceived income risks under different possible income processes seen in the macro literature. Under a clearly specified income process, I can examine to what extent the perceived income risks align with a number of benchmark predictions under full-information rational expectation(FIRE) and with a list of empirically documented facts regarding the income risk dynamics. For instance, is there a large degree of dispersion in risk perceptions among agents who the modelers assume face the same level of risks? Other questions I use the survey to answer include: are perceived risks state-dependent and counter-cyclical? Do people extrapolate and overreact to recent experiences when forming risk perceptions? Does the perceived risk reflect a reasonably good understanding of the income risks of different nature? The answers to all of these questions are yes. 

%Individuals of varying characteristics face potentially different income processes. Even under the same income process, the realizations of income differ across agents due to differences in realized shocks. In addition to the fact that realized income is not observed in these surveys, this makes it additionally challenging to undertake comparisons between perceptions and the underlying process in a similar manner as for expectations about macroeconomic variables such as inflation. A clear comparison of such spirit is also possibly sensitive to the consistency between the frequency of the reported income perception and the frequency of the underlying income process, i.e.~the time aggregation problem. Besides, I also explicitly take into account the presence of the superior information problem extensively discussed in the literature.
The most important novelty of this paper compared to previous work studying partial insurance with expectational surveys \footnote{For instance, \cite{pistaferri_superior_2001}, \cite{kaufmann_disentangling_2009}.} is that I particularly use the density survey which contains directly perceived risks.  What is special about the density survey is that agents are asked to provide histogram-type forecasts of their earning growth over the next 12 months, together with a set of expectational questions about the macroeconomy. When the individual density forecast is available, a parametric density estimation can be made to obtain the individual-specific subjective distribution. And higher moments reflecting the perceived income risks such as variance allow me
to directly characterize the perceived risk profile without relying on external estimates from cross-sectional microdata. This provides the first-hand measured perceptions on income risks that are truly relevant to individual decisions.

I established the following novel facts about the perceived income risks.

\begin{itemize}

\item People do have reasonable clues about the income risks they are facing, in the sense that heterogeneity in risk perceptions across different groups are consistent with the between-group differences in income volatility revealed from past data. For instance, younger, low-income, being female, low-education with more volatile income growth also perceive higher risks.

\item But huge heterogeneity remains. People with the same observable characteristics still show wide dispersion in risk perceptions. 

\item State-dependence.  Perceived income risks counter-cyclically react to nationwide/regional labor market conditions. 
  %I found that average perceived income risks by U.S. earners are negatively correlated with the current labor market tightness, measured by wage growth and unemployment rate. Besides, earners in states with higher unemployment rates and low wage growth also perceive income risks to be higher. This bears similarities to but important difference with a few previous studies that document the counter-cyclicality of income risks estimated by cross-sectional microdata (\cite{guvenen2014nature},\cite{catherine_countercyclical_2019}).

\item Extrapolation. People who have experienced earning volatility and unemployment have higher risk perceptions. 

\item Past-dependence. Higher experienced volatility and experience of negative labor market conditions is associated with higher perceived income risks. 
 
 
\item Decision. Perceived income risks translate into economic decisions in a way consistent with precautionary saving motives. In particular, households with higher income risk perceptions expect a higher growth in expenditure, i.e.~lower consumption today versus tomorrow.
\end{itemize}


These empirical patterns in perceived income risks have clear implications for both the level of optimal savings and cross-sectional wealth inequality via its effects on precautionary saving motives. As to the level, lower perceived risks than standard estimates imply lower precautionary savings, while the state-dependence and extrapolation both induce additional precautionary savings, as shown by \cite{caballero1990consumption}. Therefore, the quantitative implications of the survey-implied risk perceptions on the level of savings depend on the counterbalance of the two forces. The effect on wealth inequality is less ambiguous. First, allowing for heterogeneity in risk perceptions induce straightforward increase in wealth inequality simply because different risks induce different optimal savings. At the same time, there is a less direct effect via lower perceived risks. Lower perceived risks ex ante induce lower self-insurance motives, hence more ex-post wealth inequality. 

I incorporate a mechanism of subjective risk perceptions into a standard incomplete market, overlapping-generation, and general equilibrium model to quantify these effects. The objective/benchmark model blends \cite{huggett1996wealth}, the income structure of \cite{carroll1997nature}, and persistent unemployment spells and unemployment benefits, a la \cite{krueger2016macroeconomics} and \cite{carroll2017distribution}. The subjective model features a single deviation by introducing an idiosyncratic subjective state that swings between low and high-risk perceptions, the process of which is estimated by the survey data. This assumption easily accommodates heterogeneity, state-dependence, and extrapolation of risk perceptions. In comparison with the objective model, the subjective model adds a state variable to individuals' consumption problems, and its dynamics also drive the distributional evolution of the economy in wealth. I characterize the economy with a stationary equilibrium. I also explore an extension of the subjective model by assuming the risk perception state depends on employment status of the individuals. 

With the standard calibration of the model as in the literature, I \emph{expect} to show that incorporating survey-implied risk perceptions separately helps generate sizable additional wealth inequality needed to match the data. It will also shed light on if the subjective income risk profiles on average result in a higher or lower level of aggregate savings in equilibrium. As a methodological contribution, my exercises highlight the importance of introducing belief heterogeneity informed by expectational surveys into incomplete market macroeconomic models.  

%These patterns suggest that individuals have a roughly good yet imperfect understanding of their income risks. Good, in the sense that subjective perceptions are broadly consistent with the realization of cross-sectional income patterns. This is attained in my model because agents learn from past experiences, roughly as econometricians do. In contrast, subjective perceptions are imperfect in that bounded rationality prevents people from knowing about the true size and nature of income shocks as well some parameters of the process perfectly. If hardworking economists equipped with advanced econometric techniques and a large sample of income data do not necessarily specify the income process correctly, it is feasible to admit the agents in the model to be subject to the same difficulty.

%As illustrated by much empirical work of testing the rationality of expectations, it is admittedly challenging to separately account for the differences in perceptions driven by the ``truth'' and the part driven by the pure subjective heterogeneity. The most straightforward way seems to be to treat econometricians' external estimates of the income process as the proxy to the truth, for which the subjective surveys are compared. But this approach implicitly assumes that econometricians correctly specify the model of the income process and ignores the possible ``superior information'' problem that is available only to the people in the sample but not to econometricians. Both my empirical comparisons and model assumptions reconcile these possible challenges by separately characterizing the objective and subjective profile, without taking a strong stance on who is the correct and who is wrong. 

%Finally, the subjective learning model will be incorporated into an otherwise standard life-cycle consumption/saving model with uninsured idiosyncratic and aggregate risks. Experience-based learning makes income expectations and risks state-dependent when agents make dynamically optimal decisions at each point of the time. In particular, higher perceived risks will induce more precautionary saving behaviors. If this perceived risk is state-dependent on recent income changes, it will potentially shift the distribution of MPCs along income decile, therefore, amplify the channels aggregate demand responses to shocks.

\import{./}{1.1literature.tex}

