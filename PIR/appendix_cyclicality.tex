

    \hypertarget{counter-cyclicality-of-perceived-risk}{%
\subsubsection{Counter-cyclicality of perceived
risk}\label{counter-cyclicality-of-perceived-risk}}

Some studies have documented that income risks are counter-cyclical
based on cross-sectional income data.
\footnote{But they differ in exactly which moments of the income are counter-cyclical. For instance, \cite{storesletten2004cyclical} found that variances of income shocks are counter-cyclical, while \cite{guvenen2014nature} and \cite{catherine_countercyclical_2019}, in contrast, found it to be the left skewness.}
It is worth inspecting if the subjective income risk profile has a
similar pattern. Figure \ref{fig:ts_he} plots the average perceived
income risks from SCE against the YoY growth of the average hourly wage
across the United States, which shows a clear negative correlation.
Table \ref{macro_corr_he} further confirms such a counter-cyclicality by
reporting the regression coefficients of different measures of average
risks on the wage rate of different lags. All coefficients are
significantly negative.

\begin{center}
[FIGURE \ref{fig:ts_he} HERE]
\end{center}

    \begin{figure}[!ht]
      \caption{Recent Labor Market Conditions and Perceived Risks}
    \label{fig:ts_he}
    	\begin{center}\adjustimage{max size={\linewidth}}{figures/tsMean3mvrvar_he.jpg}
    \end{center}
    \begin{flushleft}Note: recent labor market outcome is measured by hourly wage growth (YoY). The 3-month moving average is plotted for both series.\end{flushleft}
    \end{figure}
    

\begin{center}
[TABLE \ref{macro_corr_he} HERE]
\end{center}

The pattern can also be seen at the state level. Table
\ref{macro_corr_he_state} reports the regression coefficients of the
monthly average perceived risk within each state on the state labor
market conditions, measured by either wage growth or the state-level
unemployment rate, respectively. It shows that a tighter labor market
(higher wage growth or a lower unemployment rate) is associated with
lower perceived income risks. Note that our sample stops in June 2019
thus not covering the outbreak of the pandemic in early 2020. The
counter-cyclicality will be very likely more salient if it includes the
current period, which was marked by catastrophic labor market
deterioration and increase market risks.

\begin{center}
[TABLE \ref{macro_corr_he_state} HERE]
\end{center}

The counter-cyclicality in subjective risk perceptions seen in the
survey may suggest the standard assumption of state-independent symmetry
in income shocks is questionable. But it may well be, alternatively,
because people's subjective reaction to the positive and negative shocks
are asymmetric even if the underlying process being symmetric. The model
to be constructed in the theoretical section explores the possible role
of both.
