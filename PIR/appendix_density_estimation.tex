
\hypertarget{density-estimation-and-variables}{%
\subsection{Density estimation of survey answers}\label{density-estimation-and-variables}}


With the histogram answers for each individual in hand, I follow
\cite{engelberg_comparing_2009} to fit each of them with a parametric
distribution accordingly for three following cases. In the first case
when there are three or more intervals filled with positive
probabilities, it was fitted with a generalized beta distribution. In
particular, if there is no open-ended bin on the left or right, then
two-parameter beta distribution is sufficient. If there is
open-ended bin with positive probability on either left or right, since the lower bound or upper
bound of the support needs to be determined, a four-parameter beta
distribution is estimated. In the second case, in which there are
exactly two adjacent intervals with positive probabilities, it is fitted
with an isosceles triangular distribution. In the third case, if there
is only one positive-probability of interval only, i.e.~equal to one, it
is fitted with a uniform distribution.

\begin{comment}
Since subjective moments such as variance is calculated based on the
estimated distribution, it is important to make sure the estimation
assumptions of the density distribution do not mechanically distort my
cross-sectional patterns of the estimated moments. This is the most
obviously seen in the tail risk measure, skewness. The assumption of log
normality of income process, common in the literature (See again
\cite{blundell_consumption_2008}), implicitly assume zero skewness,
i.e.~that the income increase and decrease from its mean are equally
likely. This may not be the case in our surveyed density for many
individuals. In order to account for this possibility, the assumed
density distribution should be flexible enough to allow for different
shapes of subjective distribution. Beta distribution fits this purpose
well. Of course, in the case of uniform and isosceles triangular
distribution, the skewness is zero by default.

Since the microdata provided in the SCE website already includes the
estimated mean, variance and IQR by the staff economists following the
exact same approach, I directly use their estimates for these moments.
At the same time, for the measure of tail-risk, i.e.~skewness, as not
provided, I use my own estimates. I also confirm that my estimates and
theirs for the first two moments are correlated with a coefficient of
0.9.

\end{comment}

For all the moment's estimates, there are inevitably extreme values.
This could be due to the idiosyncratic answers provided by the original
respondent, or some non-convergence of the numerical estimation program.
Therefore, for each moment of the analysis, I exclude top and bottom
\(1\%\) observations, leading to a sample size of around 53,180.

I also recognize what is really relevant to many economic decisions such
as consumption is real income instead of nominal income. I use the inflation expectation to convert expected nominal earning growth to real
growth expectations. 

The real earning risk, namely the variance associated with real earning growth, if we treat inflation
and nominal earning growth as two independent stochastic variables, is
equal to the summed variance of the two. The independence assumption is
admittedly an imperfect assumption because of the correlation of wage
growth and inflation at the macro level. In the Appendix, I report results of the paper with alternative assumptions about the correlation between the nominal wage and inflation expectations.
