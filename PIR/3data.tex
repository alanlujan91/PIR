
    \hypertarget{data-variables-and-density-estimation}{%
\section{Data, variables and density
estimation}\label{data-variables-and-density-estimation}}

\hypertarget{data}{%
\subsection{Data on perceived risks}\label{data}}

The data used for this paper is from the core module of Survey of
Consumer Expectation(SCE) conducted by the New York Fed, a monthly
online survey for a rotating panel of around 1,300 household heads over
the period from June 2013 to January 2020, over a total of 80 months.
This makes about 95113 household-year observations, among which around
68361 observations provide non-empty answers to the density question on
earning growth.

Particularly relevant for my purpose, the questionnaire asks each
respondent to fill perceived probabilities of their same-job-hour
earning growth to pre-defined non-overlapping bins. The question is
framed as ``suppose that 12 months from now, you are working in the
exact same {[}``main'' if Q11\textgreater1{]} job at the same place you
currently work and working the exact same number of hours. In your view,
what would you say is the percentage chance that 12 months from now:
increased by x\% or more?''.

As a special feature of the online questionnaire, the survey only moves
on to the next question if the probabilities filled in all bins add up
to one. This ensures the basic probabilistic consistency of the answers
crucial for any further analysis. Besides, the earning growth
expectation is formed for exactly the same position, same hours, and the
same location. This has two important implications for my analysis.
First, these conditions help make sure the comparability of the answers
across time and also excludes the potential changes in earnings driven
by endogenous labor supply decisions, i.e.~working for longer hours.
Empirical work estimating income risks are often based on data from
received income in which voluntary labor supply changes are inevitably
included. Our subjective measure is not subject to this problem and this
is a great advantage. Second, the earning expectations and risks
measured here are only conditional on non-separation from the current
job. It excludes either unemployment, i.e.~likely a zero earning, or an
upward movement in the job ladder, i.e.~a different earning growth rate.
Therefore, this only reflects a lower bound for the income risks facing the individuals. I will separately compare unemployment/separation expectations in Section \ref{unemployment-risk-perceptions}. 

Unemployment and other involuntary job separations are undoubtedly
important sources of income risks, but I choose to focus on the
same-job/hour earning with the recognition that individuals' income
expectations, if any, may be easier to be formed for the current
job/hour than when taking into account unemployment risks. Given the
focus of this paper being subjective perceptions, this serves as a
useful benchmark. What is more assuring is that the bias due to omission
of unemployment risk is unambiguous. We could interpret the moments of
same-job-hour earning growth as an upper bound for the level of growth
rate and a lower bound for the income risk. To put it in another way,
the expected earning growth conditional on current employment is higher
than the unconditional one, and the conditional earning risk is lower
than the unconditional one. At the same time, since SCE separately
elicits the perceived probability of losing the current job for each
respondent, I could adjust the measured labor income moments taking into
account the unemployment risk.

\hypertarget{density-estimation-and-variables}{%
\subsection{Density estimation and
variables}\label{density-estimation-and-variables}}

With the histogram answers for each individual in hand, I follow
\cite{engelberg_comparing_2009} to fit each of them with a parametric
distribution accordingly for three following cases. In the first case
when there are three or more intervals filled with positive
probabilities, it was fitted with a generalized beta distribution. In
particular, if there is no open-ended bin on the left or right, then
two-parameter beta distribution is sufficient. If there is
open-ended bin with positive probability on either left or right, since the lower bound or upper
bound of the support needs to be determined, a four-parameter beta
distribution is estimated. In the second case, in which there are
exactly two adjacent intervals with positive probabilities, it is fitted
with an isosceles triangular distribution. In the third case, if there
is only one positive-probability of interval only, i.e.~equal to one, it
is fitted with a uniform distribution.

Since subjective moments such as variance is calculated based on the
estimated distribution, it is important to make sure the estimation
assumptions of the density distribution do not mechanically distort my
cross-sectional patterns of the estimated moments. This is the most
obviously seen in the tail risk measure, skewness. The assumption of log
normality of income process, common in the literature (See again
\cite{blundell_consumption_2008}), implicitly assume zero skewness,
i.e.~that the income increase and decrease from its mean are equally
likely. This may not be the case in our surveyed density for many
individuals. In order to account for this possibility, the assumed
density distribution should be flexible enough to allow for different
shapes of subjective distribution. Beta distribution fits this purpose
well. Of course, in the case of uniform and isosceles triangular
distribution, the skewness is zero by default.

Since the microdata provided in the SCE website already includes the
estimated mean, variance and IQR by the staff economists following the
exact same approach, I directly use their estimates for these moments.
At the same time, for the measure of tail-risk, i.e.~skewness, as not
provided, I use my own estimates. I also confirm that my estimates and
theirs for the first two moments are correlated with a coefficient of
0.9.

For all the moment's estimates, there are inevitably extreme values.
This could be due to the idiosyncratic answers provided by the original
respondent, or some non-convergence of the numerical estimation program.
Therefore, for each moment of the analysis, I exclude top and bottom
\(3\%\) observations, leading to a sample size of around 53,180.

I also recognize what is really relevant to many economic decisions such
as consumption is real income instead of nominal income. I use the inflation expectation and inflation uncertainty (also estimated
from density question) to convert nominal earning growth moments to real
terms. In particular, the real
earning growth rate is expected nominal growth minus inflation
expectation of the same individuals.

\begin{eqnarray}
E_{i,t}({\Delta y}_{i,t}) = E_{i,t}(\Delta y^n_{i,t+1}) - E_{i,t}(\pi_{t+1})
\end{eqnarray}

The variance associated with real earning growth, if we treat inflation
and nominal earning growth as two independent stochastic variables, is
equal to the summed variance of the two. The independence assumption is
admittedly an imperfect assumption because of the correlation of wage
growth and inflation at the macro level. So it is should be interpreted
with caution.

\begin{eqnarray}
Var_{i,t}(\Delta y_{i,t+1}) = Var_{i,t}(\Delta y^n_{i,t+1}) + Var_{i,t}(\pi_{t+1})
\end{eqnarray}

As there are extreme values on inflation
expectations and uncertainty, I also exclude top and bottom \(2\%\) of
the observations. 

\subsection{Labor income data}

I use longitudinal data on individual labor earnings from the 2014-2017 and 2018-2020 panels of the Survey of Income and Program Participation (SIPP)\footnote{Other recent work that estimates income risks using SIPP includes \cite{bayer2019precautionary}. Different from this paper, they use quarterly total household income, instead of the monthly job-specific earning of individuals.}. Each panel of the SIPP is designed to be a nationally representative sample of the U.S. population and surveys thousands of workers. The interviews are conducted once a year to collect the individual's monthly earnings and labor market activity\footnote{This causes the ``seam'' issue well documented in the survey literature\citep{moore2008seam}., which states that cross-wave transitions are systematically larger in magnitudes than within-wave changes. Therefore, I exclude the December-to-January earning growth in estimations to address this issue.}. On average, each individual is surveyed for 33 months over the multiple waves of the survey.

For the purpose of this paper, there are obvious advantages with using SIPP over another commonly used dataset for income risk estimation, the most notable of which is the Panel Study of Income Dynamics (PSID). SIPP surveys monthly labor outcomes of workers such as earnings, hours of work, and other detailed records of job transitions, while PSID only provides biennial records of labor income for years since 1997. For the  overlapping periods between SIPP and SCE, it is possible to make a direct comparison between realized income risks at the annual frequency and the ex-ante perceptions of the earners. This is particularly crucial if income risks are time-varying and dependent upon macroeconomic conditions. Besides, given the surveyed risk perception is conditional on the same job position and hours, income risks explicitly controlling for hours of work and conditional on the same job would make the two more comparable.   
Given the earning risks expectations regarding the single job for the equal hours of work, I divide the monthly earnings from the \emph{primary job} by the average hours of work for the same job and use it to estimate risks. I restrict the SIPP sample used for estimating income risks for workers who have been staying in the same job for the entire surveyed period. In addition, I impose the following criteria. (1) only working-age population between 25-65. (2) only private-sector jobs, excluding workers from government or other public sectors. (3) no days away from work during the reference month without the pay. (4) the same job as the last year. (5) monthly wage rates that are greater than 2.5 times or smaller than 0.1 times of the average wage are excluded. This leaves me with a monthly panel of 350-1000 individual earners for the sample period 2013m3-2019m12. Appendix \ref{appendix:sipp_data} discusses the data selection and summary statistics in greater details. 